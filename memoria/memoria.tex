\documentclass[12pt]{article}
\usepackage[spanish, english, es-tabla]{babel}
\usepackage[utf8]{inputenc}
\usepackage{amsmath,amssymb}
\usepackage{graphicx}

\usepackage{eurosym}

\usepackage{array}

% text font 

\usepackage{courier}

%\usepackage{fontspec}

\usepackage[dvipsnames]{xcolor}

\usepackage{hyperref}
\usepackage{subcaption}
\usepackage[left = 2cm, right = 2cm, bottom = 2cm, top = 3cm]{geometry}

\hypersetup{
	colorlinks=true,
	linkcolor=black,
	%filecolor=magenta,      
	urlcolor=cyan,
}

\hypersetup{
	pdftitle= {Diseño y automatización de un sistema de alimentación OpenSource para animales a través de la tecnología LoRa},
	pdfauthor = {Lucía Francoso Fernández},
	pdfsubject = {Electrónica y comunicaciones móviles},
	pdfkeywords = {Arduino, LoRa, automatización, LoRaWAN, punto a punto, OpenSource}
}

% https://tex.stackexchange.com/questions/60209/how-to-add-an-extra-level-of-sections-with-headings-below-subsubsection
\newcommand{\subsubsubsection}[1]{\paragraph{#1}\mbox{}\\}
\setcounter{secnumdepth}{4}
\setcounter{tocdepth}{4}

\begin{document}
	\selectlanguage{spanish}

	\title{Diseño y automatización de un sistema de alimentación \textit{OpenSource} para animales a través de la tecnología LoRa}
	\author{Lucía Francoso Fernández}
	\date{Marzo 2021}
	
	\maketitle
	\pagebreak
	
	\tableofcontents
	
	\pagebreak

	\listoffigures
	\addcontentsline{toc}{section}{Índice de figuras}
	
	\pagebreak
	
	\listoftables
	\addcontentsline{toc}{section}{Índice de tablas}
	
	\pagebreak
	
	\section*{Agradecimientos}
	\addcontentsline{toc}{section}{Agradecimientos}
	
	\noindent [Tutor] Primero, me gustaría agradecer a mi tutor, Juan Pascual García, por aceptar este reto junto a mi, y apoyarlo en todo momento. Fue un proyecto diferente, algo arriesgado, y siempre confió en que lo sacaríamos hacia delante. Muchas gracias por tu apoyo a distancia, debido a la Covid-19, que a pesar de todo siempre he obtenido tu ayuda y tu feedback a tiempo, y confianza en todo proceso requerido en este proyecto.\\
	
	\noindent [Mi familia]  Me gustaría, además, agradecer a mi familia, en especial a mis padres, Francisco Francoso y María Dolores Fernández, ya que es difícil entender que quiera estar un año sin apenas créditos matriculados, la importancia de aprender de manera complementaria a la universidad, de buscar (y encontrar) aquello que me gusta dentro de esta carrera que, a pesar de las dificultades, siento que no podría haber elegido otra mejor. Sin ese apoyo, todo habría sido mucho más difícil, y siendo mi situación poco común, agradezco que lo hayan comprendido. A esto se suma su aportación económica, sin ella no habría podido realizar este proyecto, el cual he sentido en todo momento muy personal y muy bueno para mi, en tanto a nivel profesional, como personal como para la protectora que será beneficiaria de este dispositivo. Agradecer a mis hermanas Isabel y Claudia, que a pesar de las diferencias, y a pesar de ser la mayor, tienden a reforzarme en los momentos difíciles, recalcando mis virtudes; confiaron en este proyecto, y saben lo bueno que podría conllevar, ya que han venido innumerables veces conmigo al refugio, al igual que mi padre.\\
	
	\noindent También agradecer al resto de mi familia, que en la distancia también me apoyan. En especial a mi tía Bárbara, que siempre me llama para darme ánimos con todo lo que hago y que confía en mi desde siempre. A mis primas Sonia y María, y a mi abuela Guadalupe, las cuales son ejemplos de mujeres admirables en mi familia y a las que quiero muchísimo. Especial mención a mi abuela, Lucía Figueredo, la cual falleció el año pasado y sé que estaría muy orgullosa de ver a su nieta convertirse en ingeniera, y creando este proyecto. \\
	
	\noindent [Mis suegros] Debo agradecer también a mis suegros, Encarnación y Miguel Ángel, por dedicarme parte de su tiempo siempre que pueden, y en especial, en el desarrollo del prototipo exterior; agradezco enormemente que me hayan prestado sus herramientas, ya que sin ellas el resultado no habría sido el mismo. Muchas gracias por permitirme instalar en su casa el otro extremo de la comunicación, que ha permitido poder concebir este proyecto. \\
	
	\noindent [Magda, Clara, Pablo]. Me gustaría también agradecer a mi amiga Magdalena, mi amiga de toda la vida, la cual me apoya incondicionalmente, y me inspira en su lucha constante por mejorarse así misma y por ganarse la vida en lo que más le gusta, la música. A mi amiga Clara, la cual es una estupenda teleco, que siempre ha confiado en mi más que yo misma, y la cual me inspira por ser una persona fuerte, independiente y que tiene claro lo que quiere. A mi amigo Pablo, porque ni la distancia ni las exigentes carreras que cursamos nos han hecho perder el contacto, ni perder un vínculo que hace años se formó. Es un placer poder contar con estar tres personas.\\  
	
	\noindent [Enrique] Especial mención a mi compañero Enrique Fernández Sánchez, el cual ha sido también como mi tutor, ayudándome muchísimo a comprender la importancia del Open Source, guiándome en el conocimiento sobre microcontroladores y LoRa, y en general, aconsejándome en todo lo que ha podido y más. Gracias a él, la introducción al mundo maker se me ha hecho más sencilla, mucho más consciente y sin excesivos agobios. A esto, añadir que su aportación económica me ha ayudado muchísimo a sacar este proyecto hacia delante. Sin duda, ajeno a este proyecto, aprenderé muchas cosas de él y gracias a él, y no me cabe la menor duda de que es un gran teleco y una gran persona.\\
	
	\noindent [Patitas]  Agradecer a Patitas Unidas Los Alcázares por permitirme instalar un dispositivo en sus instalaciones, y por comprender mi ausencia en algunos meses donde el desarrollo de este proyecto era muy exigente. Especial mención a mi compañera Jen, con la que voy al refugio prácticamente siempre; su actitud y sus ganas al tratar con los perros desde luego es una cualidad a admirar, y es una inspiración que recarga fuerzas para seguir luchando por los animales.\\
	
	\noindent [Comunidad OpenSource] Por último, agradecer a todas esas personas que han subido a la nube su proyecto, su documentación y dejan todo ese desarrollo bajo licencia Open Source, permitiendo que gente como yo desarrolle su proyecto basándome en sus librerías, o me den ideas que sin duda impulsen el mío. Estarán debidamente mencionadas, ya que merecen ese crédito.\\
	
	\pagebreak
	
	\section[Introducción]{Introducción} %corchetes, marcapaginas, llaves, texto
	
	Tras finalizar los estudios de grado en ingeniería en sistemas de telecomunicaciones, se requiere como último paso para la obtención del título la elaboración del \textit{Trabajo Fin de Grado} (TFG). 
	El TFG como tal tiene unos objetivos claros, los cuales son demostrar que se han adquirido las competencias básicas intrínsecas al grado, y que el alumno es capaz de seguir aprendiendo a partir de los conocimientos ya obtenidos, innovar, desenvolverse ante un problema determinado y, en resumen, saber llevar a cabo una investigación o proyecto que tenga como objetivo la resolución de dicho problema. \\
	
	\noindent En este documento, se recoge el proyecto realizado como TFG,  el cual va en línea con la filosofía y objetivos del \textit{Trabajo Fin de Grado} en sí mismo. Se detallará el proceso de creación de un sistema de alimentación para animales automatizado, donde se monitorizan de manera remota el estado de los tanques de reserva de agua y pienso. Así, se pretende exponer el conjunto de elementos hardware y software que serán necesarios para monitorizar, automatizar y acceder a ciertos datos de forma remota empleando, principalmente, Arduino y LoRa.\\
	
	
	\subsection[Contexto y justificación del trabajo]{Contexto y justificación del trabajo}

	\noindent Este proyecto ha sido concebido con el objetivo principal de ayudar a la preservación de la vida animal, ante el aumento de especies en extinción, sobretodo en lo que llevamos de siglo \footnote{\href{https://www.nationalgeographic.com.es/naturaleza/grandes-reportajes/animales-peligro-extincion_12536}{National Geographic} , \href{https://www.bbc.com/mundo/noticias-54036796}{BBC}, \href{https://www.worldwildlife.org/descubre-wwf/historias/que-significa-especie-en-peligro-de-extincion}{WWF}, \href{https://www.fundacionaquae.org/causas-perdida-biodiversidad/}{Fundación AQUAE}}, y ante el hecho de que miles de animales domésticos siguen siendo abandonados al año en España \footnotemark. \\
	
	%Fundación Affinity, 20 minutos, La Razón, RTVE
	\footnotetext{\href{https://www.fundacion-affinity.org/observatorio/infografia-el-nunca-lo-haria-estudio-de-abandono-y-adopcion-2020}{Fundación Affinity}, \href{https://www.20minutos.es/noticia/4318383/0/el-abandono-animal-en-espana-aumenta-un-25-en-las-ultimas-semanas/}{20 minutos}, \href{https://www.larazon.es/medio-ambiente/20201118/qxv6yuokargfbnjvknn6bhm4ze.html}{La Razón}, \href{https://www.rtve.es/noticias/20200608/abandonos-animales-domesticos-se-han-disparado-espana-durante-meses-confinamiento/2015761.shtml}{RTVE}}

	\noindent Es por ello que se requiere de ayuda activa para paliar estos problemas. Las tareas de carácter solidario, en bastantes casos, no siempre cuentan con suficientes voluntarios; además, siendo un problema tan extendido y avanzado, el número de acciones que hay que llevar a cabo para aliviarlo es alto para un número limitado de voluntarios, los cuales deben incrementar considerablemente el tiempo que pasan realizando este tipo de tareas. 
	Tanto si se trata de un refugio de animales domésticos abandonados, como de reservas naturales donde se intenta repoblar una especie, los animales dependen enteramente del trabajo de los voluntarios. Así pues, es de vital importancia la optimización de las tareas de voluntariado, su automatización e, incluso, control remoto, mediante la creación de herramientas que ayuden a reducir el tiempo que se destina a tareas rutinarias para poder utilizar ese tiempo a otras tareas (de rescate, o de financiación para el mantenimiento de las instalaciones y de los propios animales, por ejemplo). \\
	
	\noindent Con el objetivo en mente de ayudar a la preservación de la fauna (y con ello, de la vida de los ecosistemas terrestres), evitando la extinción de especies y el abandono animal, se ha concebido y desarrollado este proyecto teniendo en cuenta los diferentes casos de uso (emplazamientos donde se podrá instalar el sistema, características del entorno), mejor adaptación a ellos, relación entre buenas prestaciones y bajo consumo, precio total del producto, o facilidad de uso por parte de un usuario medio. Es por ello que desde el principio se propone una serie de actuaciones a realizar, acorde a lo anteriormente mencionado, tales como:
	
	\begin{itemize}
		\item El diseño del sistema de alimentación será tal que el producto final pueda ser instalado no sólo en hogares, sino en sitios remotos, donde el acceso a recursos tales como la electricidad o Internet son escasos o inexistentes. Así pues, el dispositivo utilizará energía solar y baterías recargables, comunicación de bajo consumo y largo alcance y modo de ahorro de energía. 
		\item El dispositivo final no será pesado ni voluminoso, facilitando así tanto su transporte como su manejo. No dejará al alcance del animal electrónica, de manera que evitaremos que los animales estén en contacto con ella y posibles problemas de humedad presente en el entorno; se usarán protecciones adecuadas a la calidad de los componentes del dispositivo.
		\item El dispositivo contará con una pantalla OLED que permitirá ver a la persona que esté físicamente delante de él si funciona correctamente. También se permitirá el acceso a datos a personas interesadas que quieran consultarlos de manera remota.

	\end{itemize}

	
	\subsubsection[Ejemplo de caso de aplicación]{Ejemplo de caso de aplicación}
	
	Un ejemplo de caso de aplicación sería el entorno donde se desea situar uno de estos dispositivos para validar su funcionamiento, que en este caso se trata del refugio perteneciente a la protectora Patitas Unidas Los Alcázares, situado en el término municipal de Torre Pacheco (Murcia).
	La elección de esta ubicación se fundamenta en una serie de razones:
	
	\begin{itemize}
		\item Al ser voluntaria para esta protectora, conozco bien sus necesidades, es decir, qué puede ser de utilidad para la protectora en el refugio y qué soluciones se han probado para determinados problemas que han ido surgiendo; además, conozco bajo qué condiciones climáticas es vulnerable y qué necesidades esporádicas emergen bajo dichas condiciones. Teniendo en cuenta la zona geográfica donde se ubica (Murcia), y los años de experiencia en el refugio, se ha detectado una problemática que se manifiesta durante  periodos continuados de lluvias (lluvias torrenciales, DANA, gota fría), la cual consiste en la inundación de las zonas colindantes al refugio, inclusive carreteras de acceso, lo que impide llegar a él (cierre de carreteras, niveles altos de riesgo por precipitación, o el simple hecho de contar con grandes volúmenes de agua en la carretera que impiden la circulación segura por la vía). De
		\item Se trata de un entorno sin electricidad y sin internet. Desarrollar un proyecto y probarlo en este tipo de entorno nos ayudará a la hora de extrapolarlo a otros emplazamientos donde la ausencia de este tipo de recursos supone también una limitación y un aspecto a tener en cuenta para definir y desarrollar el proyecto en sí mismo.
		\item Se puede intentar establecer un enlace punto a punto, ya que se puede dejar fijo un equipo transmitiendo o recibiendo que, además, se conecte a internet (ya que es un recurso disponible) para subir los datos que reciba del otro extremo. Un equipo estará presente en el refugio y el otro en una casa con internet y electricidad; esto significa que al menos este extremo será más controlable, y será este extremo el que subirá datos a la nube. Nos tendremos que preocupar más del otro extremo, donde no tendremos electricidad ni internet y donde situaremos los sensores que recogerán datos y realizarán la automatización.
	\end{itemize}
	
	\subsection[Objetivos del trabajo]{Objetivos del trabajo}
	
	El objetivo fundamental del presente trabajo es el diseño de un sistema de alimentación para animales que permita la monitorización de los niveles de agua y pienso que se encuentran en depósitos de reserva, los cuales rellenan un bebedero y un comedero, respectivamente.  Con ello, se pretende automatizar el proceso de alimentación de animales, además del uso de la tecnología LoRa para tener acceso al estado del sistema de forma remota. Así pues, los objetivos concretos son:
	
	\begin{itemize}
		\item Realizar una aproximación a la tecnología LoRa.
		\item Diseñar el sistema de alimentación para animales.
		\item Realizar tanto simulaciones para el enlace LoRa que se creará entre transmisor y receptor, como cálculos teóricos que determinen si el enlace es posible.
		\item Construir el prototipo y probarlo en entorno controlado y, posteriormente, en entorno real.
		\item Crear una plataforma de representación de datos.
	\end{itemize}
	
	\subsection[Enfoque y método seguido]{Enfoque y método seguido}
	\subsection[Planificación del trabajo]{Planificación del trabajo}
	
	\noindent [Yo creo que con introducir el excel que hicimos a principios de noviembre es suficiente, ¿no? Ya que fue la planificación que hemos ido siguiendo, más o menos. Quitaría en todo caso los subapartados de Alcance, Hitos... hasta Recursos.] \\
	
	\subsubsection[Alcance]{Alcance}
	\subsubsection[Hitos]{Hitos}	
	\subsubsection[Calendario de trabajo]{Calendario de trabajo}
	\subsubsection[Tareas y diagrama de Gantt]{Tareas y diagrama de Gantt}
	\subsubsection[Riesgos e incidencias]{Riesgos e incidencias}
	\subsubsection[Recursos]{Recursos}
	\subsection[Breve sumario de productos obtenidos]{Breve sumario de productos obtenidos}
	\subsection[Breve descripción de los capítulos restantes de la memoria]{Breve descripción de los capítulos restantes de la memoria}
	
	\pagebreak
	

	\section[Estado del arte]{Estado del arte}  
	
	\noindent En este capítulo se va a exponer un análisis del estado del arte relativo al proyecto (tecnologías y técnicas necesarias para el diseño del sistema planteado). Este análisis se centra en la situación actual en tanto a  la automatización del proceso de alimentación de animales, técnicas, sistemas y proyectos similares y conceptos introductorios a Arduino y LoRa, fundamentales para poder materializar este proyecto.
	
	\subsection[Contexto actual]{Contexto actual}
	
		\noindent Existen muchos proyectos Open Source relacionados con la alimentación automática o semiautomática de animales domésticos, principalmente gatos y perros. Sin embargo, no existen proyectos que usen LoRa como tecnología radio, sino que emplean la red local WiFi del hogar donde se sitúe el dispositivo de alimentación.
	
	\noindent A pesar de ello, se ha visualizado este tipo de proyectos y tenido en cuenta para el desarrollo del prototipo en tanto a apariencia externa, comodidad de uso, eficiencia de los componentes, o, incluso, opciones para mover el pienso desde la reserva hasta el comedero del animal, o el agua desde la reserva hasta el bebedero. \\
	
	\noindent [Añadir lo que comentas sobre TFE Específico de Aprendizaje.] \\
	
	\subsection[Trabajos relacionados]{Trabajos relacionados}
	
	\noindent [No sé muy bien si eliminar esta subsección, ya que lo que son trabajos relacionados no tengo constancia de que existan. En caso de no eliminarla, ¿me podrías orientar de cara a completar este apartado?]\\

	
	\subsection[Resumen del capítulo]{Resumen del capítulo}
	
	\pagebreak
	
	\section[Diseño del sistema]{Diseño del sistema}
	\label{Sección 3: Diseño del sistema}
	\noindent A lo largo de este capítulo se pretende mostrar los procesos de concepción, análisis e investigación llevados a cabo para la creación del primer prototipo del sistema de alimentación. Para ello, primeramente, se exponen las funcionalidades que éste debe incorporar; una vez identificadas esas funcionalidades, se presta la realización de una investigación cuyo objetivo sea la búsqueda de opciones que permitan que cada funcionalidad anteriormente expuesta pueda materializarse. Para terminar el presente capítulo, se procederá a exponer las opciones que han sido escogidas, debidamente justificadas, para el diseño del prototipo provisional del sistema de alimentación. 
	
	\subsection[Funcionalidades a cubrir]{Funcionalidades a cubrir}
	
	\noindent Teniendo claros los objetivos que se persiguen con este proyecto, es posible definir las funcionalidades que ofrecería el sistema global que se pretende crear, ya que a partir de esta definición podremos empezar a deducir y establecer qué componentes son necesarios para llevar a cabo dichas funcionalidades. \\
	
	\noindent Definimos \textit{tres funcionalidades principales}: 
	
	\begin{enumerate}
		\item \textbf{Alimentación autónoma del dispositivo}
		\item \textbf{Automatización y monitorización}. Esta funcionalidad se compone de cuatro grandes bloques:
		\begin{itemize}
			\item Bloque microcontrolador
			\item Bloque comunicación radio 
			\item Bloque comedero (sensores, actuadores)
			\item Bloque bebedero (sensores, actuadores)
		\end{itemize}
	
		\item \textbf{Integridad mediante prototipo exterior}
	\end{enumerate}
	
	\subsection[Búsqueda soluciones]{Búsqueda de soluciones para cada funcionalidad}

	\subsubsection{Alimentación autónoma del dispositivo}
	\label{subseccion 3.2.1: búsqueda soluciones alimentación}
	\noindent Se entiende por alimentación autónoma la obtención de energía suficiente para el correcto funcionamiento de un dispositivo o un sistema, y que dicha obtención de energía se realice de manera cíclica e independiente o prácticamente independiente, es decir, no dependa de la acción y supervisión constante de una persona. \\
	
	\noindent Para determinar la forma en la que alimentaremos nuestro proyecto, es imprescindible conocer bien las necesidades energéticas que lo limitarán; hay que determinar, por tanto, el consumo estimado de los componentes que conforman el proyecto. También es importante saber el tamaño y/o el destino de nuestro proyecto, para determinar si estará limitado a la hora de su transporte por una fuente de alimentación pesada, o si esta no limita en absoluto; será fundamental, además, conocer la ubicación de nuestro proyecto, es decir, si estará en el exterior o interior, si el ambiente será excesivamente frío, cálido o húmedo.\\
	
	\noindent A continuación, se mencionarán algunas soluciones para alimentación autónoma. Primero, se realizará una introducción teórica a términos y definiciones que nos permitan entender su funcionamiento; después, se presentarán opciones dentro de la categoría, haciendo énfasis en sus ventajas y desventajas, y anotando para qué proyectos es recomendable su uso. Por último, se pueden mostrar datos comparativos entre las opciones dentro de la categoría para una mejor visualización de dichas diferencias.\\ 
	
	\noindent \textbf{Baterías recargables} \\
	
	\noindent \textit{Términos y definiciones} \\
	
	\noindent Las características eléctricas de una batería definen cómo actuará en el circuito global, y las características físicas tendrán un enorme impacto en el tamaño y peso global del producto al que va a alimentar. Es por ello que, a la hora de escoger una batería, hay que tener presente los requerimientos de alimentación del proyecto, ya que ésta debe ser capaz de suministrar suficiente energía para que funcione correctamente, sin que suponga una limitación sus características físicas. Algunos parámetros que hay que comprobar son la tensión nominal, la corriente, la capacidad y el material del que está hecha la batería. \\
	
	\noindent Se presenta, a continuación, una lista más extensa y explicada de los parámetros más importantes de las baterías: 
	
	\begin{itemize}
		\item  Una \textit{\textbf{celda}} es un dispositivo electroquímico capaz de suministrar la energía que resulta de una reacción química interna a un circuito eléctrico externo. \\
		Una batería se compone de una o más celdas, conectadas en paralelo o en serie para obtener la capacidad de corriente/voltaje requerida (las baterías compuestas por celdas conectadas en serie son las más comunes).
		\item La \textit{\textbf{resistencia en serie equivalente (ESR)}} es la resistencia interna presente en cualquier celda que limita la cantidad de corriente máxima que puede entregar.
		\item La \textit{\textbf{capacidad}}, medida en amperio-hora (Ah), de una batería (o celda) es su figura de mérito más importante: se define como la cantidad de corriente que una batería puede entregar durante 1 hora antes de que el voltaje de la batería llegue al final de su vida útil. 
 		\item La \textit{\textbf{tasa ``c''}}  es una corriente que es numéricamente igual a la clasificación Ah de la celda. Las corrientes de carga y descarga se expresan típicamente en fracciones o múltiplos de la tasa c. 
 		\item El \textit{\textbf{MPV (voltaje de punto medio)}} es el voltaje nominal de la celda y es el voltaje que se mide cuando la batería se ha descargado el 50\% de su energía total. \\
 		\noindent La oscilación de voltaje máxima y mínima del valor nominal es una consideración de diseño importante: una curva de descarga "más plana" significa menos variación de voltaje que el diseño debe tolerar.
 		\item El \textit{\textbf{voltaje de celda medido al final de su vida útil}} se llama \textit{\textbf{EODV}}, que significa \textit{Fin de voltaje de descarga} (algunos fabricantes se refieren a esto como \textit{EOL} o \textit{voltaje de fin de vida útil}). \\
 		\noindent Cuando se carga al máximo, el voltaje real de la celda será más alto que el MPV. Al acercarse al punto EODV (final del voltaje de descarga), el voltaje de la celda será menor que el MPV. 
 		\item La \textit{\textbf{densidad de energía gravimétrica}} de una batería es una medida de cuánta energía contiene una batería en comparación con su peso. Normalmente viene expresada en Watt-horas/kilogramo (Wh/kg).
 		\item La \textit{\textbf{densidad de energía volumétrica}} de una batería es una medida de cuánta energía contiene una batería en comparación con su volumen. Normalmente se expresa en Watt-horas/litro (Wh/l)
 		\item Un \textit{\textbf{cargador de voltaje constante}} es un circuito que recarga una batería obteniendo solo la corriente suficiente para forzar el voltaje de la batería a un valor fijo.
 		\item Un \textit{\textbf{cargador de corriente constante}} es un circuito que carga una batería al suministrar una corriente fija a la batería, independientemente del voltaje de la batería.
	\end{itemize}

	\pagebreak
	
	\begin{figure}[h]
		\begin{center}
			\includegraphics[width=0.6\textwidth]{img/chargeDischargeCurve_TxInst.png}
			\caption{Curva de la carga/descarga de una batería.}
		\end{center}
	\end{figure}


	% \noindent Entrarían dentro de esta categoría  las conocidas baterías recargables, formato LiPo o Li-ion, por ejemplo, o formato powerbank, que si bien en su interior usan baterías LiPo/Li-ion, el encapsulado en el que se presentan y la forma en que se usan son totalmente diferentes. \\
	
	\noindent \textit{Tipos de baterías recargables}\\
	
	\noindent Las principales baterías recargables que existen son: \\
	
	\begin{enumerate}
		
		\item \textit{Lithium polymer} (LiPo).  Son baterías de gran densidad de energía disponible, es decir, almacenan la mayor parte de energía para un determinado tamaño. Es por esta razón por la que los teléfonos móviles, los portátiles y otros dispositivos ligeros en peso usan baterías LiPo. Sin embargo, esta alta densidad de energía lleva consigo un coste/inconveniente, y es que son caras, y no son sólo las baterías en sí; cuando se trabaja con baterías de litio, hay que realizar una inversión para conseguir una circuitería especial que mantenga segura la batería. Por ejemplo, cuando estas cargando una batería LiPo, hay que asegurarse de que nunca se excedan los 4.2V por celda; las baterías LiPo explotarían si las sobrecargaras. No solo la carga de la batería es delicada, sino que la descarga también lo es; si se descarga una batería LiPo por debajo de 3V por celda, se perderá permanentemente parte de su capacidad, e incluso pueden deformarse/hincharse. Finalmente, hay que estar pendiente constantemente de la temperatura, para evitar que la batería se sobrecaliente. A pesar de todo esto, es común usar baterías LiPo en proyectos ligeros y pequeños, siempre y cuando se tenga un buen circuito de carga, un buen circuito de descarga (que proteja a valores bajos de tensión, haciendo de corte en el mínimo de tensión de la batería) y un buen circuito de monitorización de temperatura. 
		
		\item \textit{Lithium-ion} (Li-ion). Son muy similares a las baterías LiPo, pero a diferencia de estas, el electrodo negativo es líquido y el encapsulado y formato de presentación es totalmente diferente (las Li-ion suelen venderse sin protección y en apariencia son similares a unas pilas alcalinas AA). Una batería Li-ion almacena más potencia que las LiPo (mayor capacidad), ofrece mayores corrientes de descarga, es menos costosa de fabricar (y por tanto tienen un precio de venta más asequible),  y tiene un tiempo de vida más largo (aunque con el tiempo pierde sus propiedades). Es, al igual que las LiPo, inflamable si se daña, se calienta en exceso o se sobrecarga. El empaquetado más común es el 18650, y en el mercado existen circuitos de protección varios para evitar la sobrecarga y la sobredescarga  para este encapsulado en concreto, por lo que es una opción barata y sencilla para conseguir un prototipado rápido y realizar pruebas del mismo, siempre y cuando el proyecto no exija mucha potencia; si el proyecto exigiera más tensión de la que proporciona la batería, se puede considerar obtener más unidades y realizar una configuración en serie para aumentar la tensión, o en paralelo si lo que se desea es aumentar la capacidad, o una combinación de ambas configuraciones (usando siempre un \textit{Battery Management System} o BMS, para mantener las baterías balanceadas entre sí). Esta opción se puede considerar para baterías LiPo, pero para usuarios principiantes o poco experimentados resultará, probablemente, más complejo.
		
		\item \textit{Lithium Iron Phosphate} (LiFePO4). Son similares a las baterías LiPo, con la salvedad de que no son propensas a explosiones, como lo pueden ser las LiPo, si no se manipulan correctamente o las condiciones anteriormente mencionadas no se dan. También requieren una apropiada circuitería para cargarlas y descargarlas con cuidado, y si se comete un error se perderá permanentemente parte de la capacidad de la batería. Además, son un poco más pesadas que las baterías LiPo. Se usan normalmente en robots ligeros, herramientas eléctricas y juguetes por radio control, ya que son baterías bastante buenas entregando grandes cantidades de energía en un factor de forma pequeño y liviano.
		
		\item \textit{Sealed lead-acid} (SLA). Son una opción si se quiere comprar una batería no muy costosa, a la vez de evitar enfrentarnos a circuitos de protección de la batería. Estas baterías son bastante más duraderas que una batería de litio, y también son mejores para hacer frente a sobrecargas accidentales o sobredescargas. Tienen una excelente relación calidad-precio y también son el mejor tipo de batería frente a temperaturas extremas; es por esta razón que las baterías de plomo (o ácido-plomo) son la opción estándar para coches, motos, almacenamiento de energía solar y fuentes de alimentación de emergencia. El inconveniente de usar baterías de plomo es que son grandes y pesadas. No es recomendable, por tanto, usar baterías de plomo en aplicaciones portátiles.
		
		\item Níquel-cadmio (\textit{Nickel-cadmium}, NiCd). Los dispositivos portátiles, como los primeros teléfonos móviles, usaban baterías de níquel-cadmio. Son baterías baratas, que en teoría duran más tiempo que las baterías SLA; sin embargo, sufren el llamado efecto memoria, por lo que hay que regularmente descargarlas por completo, luego recargarlas por completo para mantener una alta capacidad, ya que, de lo contrario, estaríamos perdiendo vida útil de la batería. Así pues, en la actualidad no se suele usar este tipo de baterías, aunque se siguen vendiendo.
		
		\item \textit{Nickel Metal Hydride} (NiMH). Las baterías de níquel-metalhidruro o de níquel hidruro metálico tienen una mayor densidad de energía que las baterías de níquel-cadmio, en tanto a precio no son mucho más costosas y no sufren el efecto memoria. Son más grandes y pesadas que las baterías de litio, pero es bastante seguro trabajar con ellas, por lo que son las baterías más usadas por el consumidor medio en términos de pilas recargables (las encontramos en cualquier supermercado, y cargarlas es barato y simple). Hay que tener en cuenta que las baterías de NiMH, por norma general, tienen un ratio de autodescarga elevado, lo que quiere decir que a pesar de que no se esté haciendo uso de la batería, se autodescargan igualmente en un par de meses, lo que hace que no sean una buena opción para objetos de uso cotidiano (como mandos a distancia). Se recomienda que, si se necesita hacer uso de este tipo de baterías, se adquieran con bajo ratio de autodescarga (como Panasonic Eneloop), las cuales pueden resultar más baratas a largo plazo que comprar unas pilas alcalinas.
		
	\end{enumerate}
	
	\noindent \textit{Comparativa entre tipos de baterías recargables} \\
	
	\noindent \textit{\textbf{Densidad de energía}}
	
		\begin{figure}[h]
		\begin{center}
			\includegraphics[width=0.7\textwidth]{img/energyDensityComparison_TxInst.png}
			\caption{Comparación densidad de energía para diferentes baterías.}
			\label{fig: comparación densidad energía}
		\end{center}
	\end{figure}
	
	 
	
	\noindent Al revisar los datos de la Figura \ref{fig: comparación densidad energía}, la ventaja de Li-Ion en densidad gravimétrica es claramente la más sorprendente, casi duplicando las cifras de Ni-Cd y Ni-MH. Esto significa que los productos que funcionan con celdas de iones de litio se pueden hacer mucho más livianos sin sacrificar el tiempo de ejecución. Alternativamente, si el peso de la batería se mantiene igual, el tiempo de funcionamiento se duplicará si se utilizan baterías de iones de litio. Este hecho explica la razón por la que Li-Ion está reemplazando rápidamente al Ni-MH en los teléfonos celulares y computadoras portátiles de primera línea. \\
	
	\noindent \textit{\textbf{Estabilidad de tensión o tensión de celda}}\\
	
	\noindent La tensión proporcionada para alimentar la carga es la más importante: las baterías Ni-Cd y Ni-MH tienen una tensión por celda de 1.25V (sus tensiones de descarga se asumen por norma general idénticas). \\
	\noindent La tensión por celda de las baterías Ni-Cd y Ni-MH son sólo una tercera parte de la tensión nominal de 3.6V proporcionada por una celda de Li-ion (Figura \ref{fig: curva descarga celdas}), lo que significa que se requerirían tres celdas de Ni-Cd/Ni-MH conectadas en serie para igualar la tensión de una única celda de Li-ion.\\
	\noindent Sin embargo, la Figura \ref{fig: curva descarga celdas} muestra también la gran ventaja de las baterías Ni-Cd y Ni-MH: su curva de descarga es extremadamente plana, próxima a la de una batería ideal. Esta importante diferencia entre tipos de baterias significa que las celdas de Ni-Cd y Ni-MH son adecuadas para su uso con reguladores lineales, pero las baterías de Li-ion requieren convertidores reductores (\textit{Buck converters}) para obtener una buena eficiencia de conversión de energía en la fuente de alimentación. \\
		
	\begin{figure}[h]
		\begin{center}
			\includegraphics[width=0.7\textwidth]{img/cellDischargeCurve_TxInst.png}
			\caption{Curva de descarga de una celda Li-ion y una celda Ni-Cd/Ni-MH.}
			\label{fig: curva descarga celdas}
		\end{center}
	\end{figure}
	
	\pagebreak 
	
	\noindent \textit{\textbf{Corriente pico}}\\	
	
	\noindent \textit{\textbf{Autodescarga}}\\	
	
	\noindent \textit{\textbf{Tiempo de carga}}\\	
	
	\noindent \textit{\textbf{Coste}}\\
	
	\noindent \textit{\textbf{Fiabilidad}}\\		
	
	\noindent \textit{\textbf{Modos de falla relacionados con la edad}}\\	
	
	\noindent \textit{\textbf{Temperatura de funcionamiento}}\\	
	
	\noindent \textbf{Panel solar}\\ 
	
	\noindent Un panel solar, por si solo, no realiza la tarea de alimentación autónoma. Requiere, como mínimo, de una batería donde almacenar la energía que captan, por lo que, necesariamente, habría que escoger entre algunas de las opciones planteadas en el apartado anterior. Además, una batería no irá conectada directamente al panel solar, ya que se requiere un cargador solar que haga de intermediario, por una parte, entre la batería y el panel, y por otra parte, entre la batería y el dispositivo al que alimenta dicha batería. \\
	
	\noindent Aunque esto será explicado con más detalle más adelante, se considera necesario establecer desde un principio los requerimientos mínimos para poder usar un panel solar de cara a la funcionalidad de alimentación autónoma, ya que dependen del apartado que acabamos de presentar sobre baterías recargables, y estos requerimientos mínimos establecerán una base de cara a la explicación sobre sistemas fotovoltaicos más complejos. \\
	
	\noindent \textit{Funcionamiento y composición de un panel solar}
	
	\begin{figure}[h]
		\begin{center}
			\includegraphics[width=0.6\textwidth]{img/layersSolarPanel.png}
			\caption{Capas y componentes que componen un panel solar.}
			\label{fig: capas panel solar}
		\end{center}
	\end{figure}

	\pagebreak
	
	\noindent Un panel solar es una agrupación de muchas células fotovoltaicas (PV) que están cubiertas con vidrio protector y unidas con un marco de metal. Es por eso que el nombre oficial de un panel solar es "módulo fotovoltaico" (\textit{PV module}). Estas células solares fotovoltaicas están hechas de material semiconductor, generalmente silicio, que se corta en tiras muy finas.\\
	
	\noindent Los paneles solares tienen muchas capas. La capa de células fotovoltaicas es donde se produce la electricidad. Otras capas, como las capas de vidrio y encapsulante, están ahí para proteger las células fotovoltaicas para que los módulos puedan producir electricidad de manera adecuada.\\
	
	\noindent Cada célula fotovoltaica tiene una capa negativa y una capa positiva. La capa negativa tiene electrones adicionales y la capa positiva tiene \textit{huecos} para esos electrones. La electricidad significa electrones en movimiento, por lo que para que un panel solar genere electricidad solo necesitamos algo de energía para hacer que esos electrones se suelten y fluyan de la capa negativa a la capa positiva. \\
	
	\noindent Así pues, en resumen, la energía de los fotones del sol hace que los electrones en el lado negativo de la celda fotovoltaica se muevan (existe electricidad). Esto se llama \texttt{efecto fotovoltaico}.\\
	
	\begin{figure}[h]
		\begin{center}
			\includegraphics[width=0.6\textwidth]{img/photovoltaic_effect.png}
			\caption{Dibujo representativo del efecto fotovoltaico.}
			\label{fig: efecto fotovoltaico}
		\end{center}
	\end{figure}
	
	\pagebreak
	
	\noindent Los electrones necesitan un camino a seguir y la electricidad debe estar en una forma útil. Es por ello que la ruta que creemos para los electrones (el circuito) es importante, ya que, cuando salen de la capa negativa de la celda fotovoltaica, queremos que fluyan a través de nuestras cargas (como nuestras luces y electrodomésticos, en el ejemplo del uso doméstico) para que los electrones puedan dar energía a esas cargas a medida que avanzan hacia la capa positiva de la celda fotovoltaica. \\
	
	\noindent \textit{\textbf{Uso en hogares de energía solar: meter aquí explicación sistema fotovoltaico}} \\
	
	\noindent Los paneles solares producen electricidad de corriente continua (CC, DC), pero la generación de electricidad que usamos en nuestros hogares es de corriente alterna (CA, AC). Para solucionar este problema, agregamos un inversor a nuestro circuito. Luego, el inversor convierte la energía de CC en energía de CA. \\
	
	\noindent El inversor es un factor clave para determinar cómo funcionan los paneles solares porque sin un inversor en nuestro sistema fotovoltaico, no podemos hacer mucho con la energía generada por los paneles solares. Esto último no aplica a nuestro caso, donde nos interesa crear un proyecto que se alimente de manera autónoma, sin uso de la electricidad disponible en el hogar; el panel solar, en este caso, recargaría una pila, y por ello necesitamos la energía DC, y no necesitamos usar un inversor a AC. \\ 
	
	\noindent \textit{Parámetros}\\
	
	\noindent Los principales paramétros que se usan para caracterizar el rendimiento de las células solares son la potencia máxima $P_{max}$ (\textit{peak power}), la densidad de corriente en cortocircuito $J_{sc}$, la tensión en circuito abierto $V_{oc}$, y el factor de forma \textit{FF}. Estos parámetros se determinan a partir de la curva J-V característica cuando la célula está iluminada, tal como se muestra en la Figura \ref{fig: curva J-V célula solar}. La eficiencia de conversión $\eta$ se puede determinar a partir de estos parámetros. \\
	
	\noindent Nótese que se menciona específicamente \textit{parámetros de una célula solar}; un panel solar, como se ha mencionado anteriormente, consta de varias células fotovoltaicas, por lo que los parámetros especificados, en términos globales y tal como se construyen los paneles solares, influyen en las especificaciones finales del panel según el número de células del que conste (ver Figura \ref{fig: solar panel of identical solar cells}). \\
		\begin{figure}[h]
		\begin{subfigure}{0.5\textwidth}
			\includegraphics[width=1\textwidth]{img/JV_curve_solarCell.png} 
			\caption{Determinación de los parámetros de una célula solar a partir de su curva J-V.}
			\label{fig: curva J-V célula solar}
		\end{subfigure}
		\begin{subfigure}{0.5\textwidth}
			\includegraphics[width=1\linewidth]{img/IV_curve_solarCell.png}
			\caption{Curva I-V de una célula solar.}
			\label{fig: I-V curve of a solar cell}
		\end{subfigure}
		\caption{Curvas J-V e I-V de una célula fotovoltaica}
		\label{fig:curvas J-V e I-V célula PV}
	\end{figure}

	\begin{figure}[h]
		\begin{center}
			\includegraphics[width=0.7\textwidth]{img/effectTwoIdenticalCells.png}
			\caption{Comportamiento de un panel solar formado por dos células solares idénticas.}
			\label{fig: solar panel of identical solar cells}
		\end{center}
	\end{figure}
	

	\noindent Para realizar medidas fiables y definir la curva característica J-V, se deben realizar las medidas bajo condiciones estándar de pruebas (\textit{standard test conditions, STC}); estas son, una temperatura constante a $25^{\circ}C$, una irradiancia total de 1000 $W/ m^2$ y un espectro solar de AM1.5 (sol directo y con ángulo cenital de $48,2 ^{\circ}$). Esto lo tendremos en cuenta si queremos comprobar que las especificaciones del producto son correctas, o por el contrario el producto viene defectuoso. En este proyecto no es prioritario el estudio profundo y comprobación exhaustiva del panel solar, por lo que no se realizarán dichas pruebas; sin embargo, sí realizaremos un estudio teórico de los parámetros que definen a una célula solar, como también lo hemos hecho de los materiales que lo componen y su funcionamiento general.\\
	
	
	\noindent \textit{\textbf{Densidad de corriente en cortocircuito}} \\
	
	\noindent La corriente en cortocircuito $I_{sc}$ es la corriente que fluye a través de un circuito externo cuando los electrodos de una célula solar se cortocircuitan. La corriente en cortocircuito de una célula solar depende de la incidencia del flujo de fotones sobre ella, la cual está determinada por el espectro de la luz incidente. También depende del área de la célula solar. Para eliminar la dependencia del área de la célula solar sobre $I_{sc}$, normalmente se usa la densidad de corriente en cortocircuito para describir la máxima corriente entregada por una célula solar. La máxima corriente que una célula solar puede entregar depende mucho de las propiedades ópticas de la célula solar, tales como la absorción en la capa absorbente y la reflexión. \\
	
	\noindent Las células solares de silicio cristalino pueden entregar, bajo un espectro AM1.5, una posible densidad de corriente máxima de 46 $mA/cm^2$. En las células solares de c-Si de laboratorio, las medidas de $J_{sc}$ están por encima de 42 $mA/cm^2$, mientras que las células solares comerciales tienen un $J_{sc}$ que excede los 35 $mA/cm^2$. \\
	
	\noindent \textit{\textbf{Tensión en circuito abierto}} \\
	
	\noindent La tensión en circuito abierto es la tensión en la cual ninguna corriente fluye a través de un circuito externo. Es la tensión máxima que una célula solar puede entregar. $V_{oc}$ corresponde a la tensión de polarización directa, a la cual la densidad de corriente de oscuridad compensa la densidad de corriente fotovoltaica (\textit{fotocorriente}). Depende de la densidad de corriente de saturación de la célula solar y de la corriente fotogenerada. Mientras ésta última varía poco, la corriente de saturación puede variar en órdenes de magnitud. La corriente de saturacón depende de la recombinación en la célula solar, por lo que $V_{oc}$ es una medida de la cantidad de recombinación de la célula solar.\\
	
	\noindent Las células solares de laboratorio hechas de silicio cristalino dan valores de $V_{oc}$ de hasta 720 mV bajo condiciones AM1.5 estándar, mientras que las células solares comerciales normalmente dan valores de $V_{oc}$ por encima de los 600 mV. \\
	
	\noindent \textit{\textbf{Factor de forma}} \\
	
	\noindent El factor de forma es la relación entre la potencia máxima ($P_{max} = J_{mpp} V_{mpp}$) generada por una célula solar y el producto de $V_{oc}$ con $J_{sc}$. \\
	
	\noindent El subíndice \textit{mpp} hace referencia al \textit{punto de potencia máxima} (\textit{maximum power point, MPP}) de la célula solar, es decir, el punto de la curva característica J-V de la célula solar al cual la célula solar ofrece su máxima potencia de salida.  Para optimizar la operación de los sistemas fotovoltaicos, es muy importante que las células solares (o módulos fotovoltaicos) operen sobre el MPP. Esto se puede asegurar con un seguidor de punto de máxima potencia (\textit{maximum power point tracking, MPPT}). Se trata de un algoritmo, el cual se usa en controladores de carga solar para extraer la máxima potencia disponible del módulo fotovoltaico bajo ciertas condiciones; dichas condiciones (se entrará más en detalle más adelante), como son la radiación solar a lo largo del día y año, la temperatura ambiente y la temperatura de la célula, afectan a la potencia máxima de salida de la misma, y es entonces cuando un MPPT actúa, haciendo operar a la célula solar al valor de tensión más eficiente, el MPP. El controlador de carga mantiene la tensión y la corriente a un nivel optimizado, donde el módulo fotovoltaico brinda la mayor cantidad de energía a la carga; es ``consciente'' del estado de carga de la batería y actúa en consecuencia. \\
	
	\noindent El uso de un seguidor de punto de máxima potencia, sin embargo, encarece el costo total de un proyecto; si bien para aplicaciones que requieran la extracción de la máxima potencia de un panel solar (por limitaciones de espacio, por ejemplo, donde deberíamos explotar las características de los paneles que pudiéramos colocar), en el proyecto que estamos diseñando no es el caso, por lo que no usaremos herramientas de este estilo. Además, aunque pueden encontrarse o construirse de manera aislada (se trata de un conversor DC a DC), también suelen encontrarse incorporados a inversores, los cuales tampoco son necesarios, ya que no es energía solar para uso en hogar. Como anotación, existen otros controladores de carga que no son MPPT, como el PWM o \textit{shunt controller} (regulador de carga tipo paralelo o de desviación). \\
	
	\begin{figure}[h]
		\begin{center}
			\includegraphics[width=0.7\textwidth]{img/mppt_curve.png}
			\caption{Curva P-V de una célula solar, indicando el punto MPP.}
			\label{fig: MPPT in P-V curve of a solar cell}
		\end{center}
	\end{figure}
	
	\noindent Por último, anotar que el factor de forma no varía tanto con $V_{oc}$, sino más con el material del que están hechas las células solares. \\
	

	\noindent \textit{\textbf{Eficiencia de conversión}} \\
	
	\noindent La eficiencia de conversión se calcula como la relación entre la potencia potencia generada y la potencia incidente.  Como vemos en la siguiente fórmula, y tal como establece su propia definición, depende de todos los parámetros anteriormente descritos: \\
	
	\begin{equation}
		 \eta = \frac{P_{max}}{I_{in}} = \frac{J_{mpp} V_{mpp}}{I_{in}} = \frac{J_{sc} V_{oc} FF}{I_{in}} 
	\end{equation}
	
	\noindent Los parámetros externos típicos de una célula fotovoltaica de silicio cristalino son $J_{sc} \approx 35 mA/cm^2$, $V_{oc}$ hasta 0.65V y FF en el rango entre 0.75 y 0.80; esto implica que la eficiencia de conversión caiga en el rango entre 17 y 18 \%. \\
	
	\noindent \textit{\textbf{Circuito equivalente}} \\
	
	\noindent Una célula fotovoltaica iluminada puede comportarse como un diodo ideal, y este comportamiento puede describirse a través de un circuito equivalente simple, como el que se muestra en la Figura \ref{fig: equivalent circuit solar cell} (a). donde un diodo y una fuente de corriente están conectados en paralelo. El diodo está formado por una unión p-n. \\
	
	\begin{figure}[h]
		\begin{center}
			\includegraphics[width=0.5\textwidth]{img/circuitEquivalent_solarCell.png}
			\caption{Circuito equivalente de una célula fotovoltaica, en el caso ideal (a) y considerando pérdidas (b).}
			\label{fig: equivalent circuit solar cell}
		\end{center}
	\end{figure}
	
	\pagebreak
	
	\begin{figure}[h]
		\begin{center}
			\includegraphics[width=0.6\textwidth]{img/JV_curve_withRsRp.png}
			\caption{Curva J-V de una célula fotovoltaica, mostrando el efecto de aumentar o disminuir el valor de las resistencias $R_{s}$ y $R_{p}$ o $R_{sh}$ (\textit{shunt}).}
			\label{fig: JV curve with non-ideal model (one diode)}
		\end{center}
	\end{figure}

	\noindent Sin embargo, en la práctica, el factor de forma está influenciado por una resistencia en serie $R_{s}$, y una resistencia de derivación $R_{p}$. La influencia de estos parámetros en la curva característica J-V de una célula solar puede estudiarse usando el circuito equivalente mostrado en la Figura \ref{fig: equivalent circuit solar cell} (b). Además, en células solares reales, el factor de forma también está influenciado por una recombinación adicional que ocurre en la unión p-n. Este diodo no ideal se representa normalmente con un circuito equivalente formado por dos diodos, uno ideal con un factor de idealidad\footnotemark \enspace igual a la unidad, y otro diodo no ideal con un factor de idealidad mayor que uno. \\
	
	\footnotetext{El factor de idealidad del diodo es una medida de cómo de cerca el diodo sigue la ecuación del diodo ideal.}

	
	\begin{figure}[h]
		\begin{center}
			\includegraphics[width=0.5\textwidth]{img/nonIdeal_equivCircuit_solarCell.png}
			\caption{Circuito equivalente de una célula fotovoltaica con un modelo de dos diodos, donde $n_{1}$ y $n_{2}$ son los factores de idealidad e $I_{d1}$ y $I_{d2}$ las corrientes que atraviesan cada diodo.}
			\label{fig: non-Ideal equivalent circuit solar cell}
		\end{center}
	\end{figure}	
	
%	\noindent Uno de los parámetros más importantes en un panel solar es la \texttt{eficiencia}. Mide el porcentaje de luz solar que llega a la celda solar y que se convierte en electricidad utilizable. Cuanto mayor sea la eficiencia, menor será la superficie que necesitarán los paneles solares para satisfacer sus necesidades energéticas, pero mayor será el precio. \\
	
%	\noindent Algunas personas asocian mayores niveles de eficiencia con paneles de mayor calidad, pero esto no es necesariamente cierto. Alta eficiencia significa que su sistema fotovoltaico solar utilizará menos espacio en su azotea. Por lo tanto, quienes viven en áreas de alta densidad y tienen un espacio muy limitado para la energía solar generalmente están más preocupados por los niveles de eficiencia. \\
	
	\noindent \textit{Factores principales que influyen en la potencia de salida de una célula PV}\\
	
	\noindent \textit{\textbf{Radiación solar}}
	\begin{figure}[h]
		\begin{subfigure}{0.45\textwidth}
			\includegraphics[width=0.8\textwidth]{img/generalEffect_irradianceIVcurve_solarCell.png} 
			\caption{Efecto general.}
			\label{fig: variación general en la curva I-V de una célula PV con la intensidad de la radiación solar.}
		\end{subfigure}
		\begin{subfigure}{0.5\textwidth}
			\includegraphics[width=1.05\linewidth]{img/concreteEffect_irradianceIVcurve_solarCell.png}
			\caption{Efecto según valores concretos de radiación.}
			\label{fig:variación con valores en la curva I-V de una célula PV con la intensidad de la radiación solar.}
		\end{subfigure}
		\caption{Variación en la curva I-V ante la radiación solar.}
		\label{fig:variación I-V radiación solar}
	\end{figure}
	
	\pagebreak
	
	\noindent \textit{\textbf{Temperatura ambiente y temperatura de la célula}} \\
	
	\begin{figure}[h]
		\begin{subfigure}{0.45\textwidth}
			\includegraphics[width=0.8\textwidth]{img/generalEffect_celTempIVcurve_solarCell.png} 
			\caption{Efecto general.}
			\label{fig: variación general en la curva I-V de una célula PV con la temperatura.}
		\end{subfigure}
		\begin{subfigure}{0.55\textwidth}
			\includegraphics[width=1\linewidth]{img/concreteEffect_TempIVcurve_solarCell.png}
			\caption{Efecto según valores concretos de temperatura ambiente.}
			\label{fig:variación con valores en la curva I-V de una célula PV con la temperatura.}
		\end{subfigure}
		\caption{Variación en la curva I-V ante la temperatura ambiente}
		\label{fig:variación I-V temperatura}
	\end{figure}
	
	\noindent Nótese que en la Figura \ref{fig: variación general en la curva I-V de una célula PV con la temperatura.} se hace referencia a \textit{temperatura de la célula} y en la Figura \ref{fig:variación con valores en la curva I-V de una célula PV con la temperatura.} a \textit{temperatura ambiente}. Si bien son dos magnitudes diferentes, la temperatura ambiente tiene un efecto sobre la temperatura de la célula que puede considerarse lineal. A una temperatura ambiente de $25^{\circ}C$, la temperatura de una célula de silicio cristalino suele situarse en el valor de $48^{\circ}C$. \\
	
	\begin{figure}[h]
		\begin{subfigure}{0.52\textwidth}
			\includegraphics[width=1\textwidth]{img/PVcellTemp_AirTemp.png} 
			\caption{}
			\label{fig: solar cell temp vs air temp.}
		\end{subfigure}
		\begin{subfigure}{0.55\textwidth}
			\includegraphics[width=1\linewidth]{img/powerPVCell_airTemp.png}
			\caption{}
			\label{fig:power output of a solar cell vs air temp.}
		\end{subfigure}
		\caption{Relación entre temperatura ambiente, radiación solar, con la potencia de salida de una célula PV (\ref{fig: solar cell temp vs air temp.}) y temperatura de ésta (\ref{fig:power output of a solar cell vs air temp.}).}
		\label{fig:solar temp}
	\end{figure}
	
	\pagebreak
	
	\noindent \href{https://www.qpvgroup.org/blog/2019/2/3/how-does-air-temperature-affect-photovoltaic-solar-panel-output}{LEER ESTO Y COMENTAR POR AQUÍ, inclusive fórmula porque es relación lineal entre temperaturas ambiente y célula} \\
	
	\begin{figure}[h]
		\begin{center}
			\includegraphics[width=0.7\textwidth]{img/cellTemp_ambTemp_Irrad_Pow.png}
			\caption{Variación de la potencia de salida de una célula PV dependiendo de la hora del día, la temperatura y la radiación solar.}
			\label{fig:cellTemp_ambTemp_Irrad_Pow}
		\end{center}
	\end{figure}
	
	\pagebreak
	
	\noindent \textit{Opciones: tipos de paneles solares}\\
	
	\noindent Existen diferentes tipos de células fotovoltaicas disponibles en el mercado, aunque en su mayoría están fabricadas de silicio (Si), el segundo elemento más abundante en la corteza terrestre (después del oxígeno), el cual es el mismo material del cual está hecha la arena. El silicio por sí mismo no es muy puro y debe ser refinado para conseguir una pureza de hasta el 99.99 \% antes de que pueda usarse como material semiconductor en muchos tipos distintos de células fotovoltaicas, transistores o circuitos integrados digitales. El uso de silicio en la fabricación de células fotovoltaicas produce la típica célula fotovoltaica de color azul uniforme. \\
	\noindent Los dos tipos principales de materiales usados para células fotovoltaicas son el silicio cristalino y los depósitos de película fina, los cuales varían entre unos y otros en términos de eficiencia en absorción de luz, eficiencia en la conversión de energía, tecnología de fabricación y coste de producción. Las células fotovoltaicas (a partir de ahora, células PV, \textit{PhotoVoltaic}) hechas de silicio cristalino son el tipo más común de células PV que se utilizan en la actualidad y también son uno de los primeros dispositivos PV que tuvieron éxito.\\
	
	\noindent Los tres tipos generales de células PV hechas de silicio son:\\ 
	
	\begin{itemize}
		\item Silicio monocristalino, conocido tambien como silicio de un único cristal, cristal de silicio único o silicio monocristal (\textit{Mono-crystalline silicon, single-crystal silicon}).
		\item Silicio policristalino, conocido tambien como silicio de varios cristales, silicio multicristalino o silicio multicristal (\textit{Poly-crystalline silicon, multi-crystal silicon}).
		\item Silicio de película fina. \\
	\end{itemize}

	\noindent \textit{\textbf{Silicio cristalino (c-Si)}} \\
	
	\noindent \textit{\textbf{Célula PV de película fina}} \\
	
	\noindent \textit{\textbf{Otros tipos de células PV}} \\
	
	\subsubsection{Automatización y monitorización}
	\label{subsección 3.2.2: automatización y monitorización}
	
	\noindent En este apartado, se mostrará una visión de conjunto acerca de aquellos elementos mediante los cuales consigamos el objetivo de automatizar y monitorizar el sistema de alimentación. Se presentará el resultado de investigar sobre opciones que puedan ayudar a la consecución de los objetivos, tanto en la parte de microcontrolador como en la parte de comunicación radio, y demás actuadores y sensores en las partes de bebedero y comedero. Hay que tener presente que en todo momento se estarán buscando soluciones \textit{Open Source}, ya que se planteó como objetivo y filosofía a seguir en este proyecto.\\
	
	\subsubsubsection{Microcontrolador}
	
	\noindent \textit{Disclaimer:} a lo largo de este apartado y consecutivos se hablará de la palabra microprocesador y microcontrolador. Si bien no tienen el mismo significado, se usará la palabra microcontrolador para determinar el conjunto global de una placa cuyo corazón sea un microcontrolador, de manera que podamos aligerar un poco la literatura y lectura de este documento. Un microprocesador es la CPU, mientras que un microcontrolador compone un sistema embebido en un único \textit{chip} formado por memoria, procesador y entradas/salidas programables. Por ejemplo, llamaremos microcontrolador al arduino nano o uno porque se basa en el microcontrolador ATmega328.\\
	
	\noindent En tanto a microcontroladores, la opción \textit{Open Source} más conocida seguramente sea Arduino. Sin embargo, no significa que no existan más opciones en tanto a microcontroladores. Es importante recordar que ni Arduino ni las alternativas que se van a presentar son ordenadores, son microcontroladores, esto es, circuitos integrados diseñados para llevar a cabo una tarea única de manera cíclica (una vez que programado para una tarea, ejecutará esa tarea y sólo esa tarea una y otra vez, dependiendo de cómo haya sido programado).\\
	
	\noindent A continuación, se va a presentar una lista con algunas opciones de microcontroladores. Vale la pena señalar que la mayoría de estas placas tienen variaciones que también vale la pena revisar; la elección de un microcontrolador u otro dependerá de lo que necesite nuestro proyecto (si necesita muchas entradas porque tendremos muchos periféricos, si necesitamos ahorro de energía al máximo, o una buena potencia de procesamiento, etc). \\
	
	\begin{itemize}
		\item \textbf{Arduino}. Se trata de una plataforma de desarrollo Open Source hardware y software, característica por su facilidad de uso e integración entre la solución hardware (placas Arduino) y la solución software (Arduino IDE). Existe una gama variada de placas Arduino, entre las que podemos destacar la nano o la uno. A modo de ejemplo, se mostrará la información de la uno, placa recomendada por la propia organización Arduino para principiantes.\\
		
		\textit{Especificaciones:} ATMega328P a 16 MHz, 32 kB de memoria flash, 2kB de SRAM y 1kB de EEPROM.\\
		
		\textit{Características:} 14 pines I/O digitales, 6 entradas analógicas, conexión USB, conector jack. \\
		
		\textit{Precio:} 20 \euro \enspace en Arduino.\\
		
		\item \textbf{NodeMCU}. NodeMCU (\textit{Node MicroController Unit}) es un entorno de desarrollo de software y hardware de código abierto que se basa en un sistema en chip (SoC) muy económico llamado ESP8266. El ESP8266, diseñado y fabricado por Espressif Systems, contiene todos los elementos cruciales de un ordenador: CPU, RAM, redes (wifi) e incluso un sistema operativo y SDK modernos. Cuando se compra por separado, el chip ESP8266 cuesta solo 2\$ por unidad. Esto lo convierte en una buena opción para proyectos de IoT de todo tipo. \\
		
		\textit{Especificaciones:} SoC basado en ESP8266 a 80 MHz, 64 kB de SRAM, 4 MB de memoria flash.\\ 
		
		\textit{Características:} compatibilidad con Arduino IDE, 16 GPIO, USB-TTL en CP2102 incluido onboard, tamaño reducido, puerto micro-USB, WiFi.\\
		
		\textit{Precio:} aprox. 8 \euro \enspace en Amazon.\\
	
		\item \textbf{STM32}. La familia STM32 de microcontroladores de 32 bits basada en el procesador Arm Cortex-M ofrece productos que combinan un rendimiento muy alto, capacidades en tiempo real, procesamiento de señales digitales, operación de bajo consumo / bajo voltaje y conectividad, mientras se mantiene una integración completa y facilidad de desarrollo.
		
		La gama de microcontroladores STM32, basada en un núcleo estándar de la industria, viene con una amplia variedad de herramientas y software para respaldar el desarrollo de proyectos, lo que hace que esta familia de productos sea ideal tanto para proyectos pequeños como para plataformas de extremo a extremo. \\
		
		\textit{Especificaciones:} depende del microcontrolador. Ejemplo de STM32F103C876, 72MHz Cortex-M3, SRAM de 20kB, 64/128 kB de memoria flash. \\
		
		\textit{Características:} software propio pero compatible con Arduino IDE, tamaño similar al arduino nano, dependiendo del microcontrolador tendremos diferentes tipos dentro de la gama STM32 (bajo consumo, altas prestaciones, conectividad, mainstream). \\
		
		\textit{Precio:} depende del microcontrolador. En este ejemplo, aprox. 10\$ en eBay. \\
		
		\item \textbf{The Pinguino 45K50}. Pinguino Project existe para acercar la simplicidad del lenguaje Arduino a los microcontroladores PIC de Microchip pero con un hardware USB incorporado y una placa que se puede construir en casa. El 45K50 es una de las opciones que nace de este proyecto Open Source.  Viene como un kit que se debe ensamblar antes de usarlo. \\
		
		\textit{Especificaciones:} procesador de 12 MIPS de 8 bits que se ejecuta a 48 MHz.\\
		
		\textit{Características:} 17 entradas / salidas digitales con 5 entradas analógicas compartidas, 2 salidas PWM, UART para comunicación en serie, etc.  IDE propio, también Open Source, compatible con el lenguaje Arduino y sus librerías.\\
		
		\textit{Precio:} aprox. 25\$.\\
		
		\item \textbf{Nanode}. Nanode ofrece una amplia gama de microcontroladores. Viene como un kit de desarrollo, lo que significa se debe ensamblar y soldar previamente antes de comenzar con la placa. 	El microcontrolador Nanode puede servir como alternativa a Arduino Uno, Mega y Yun, y también es Open Source.
		
		\textit{Especificaciones:} ATMega328P a 16MHz, 2kB de SRAM, 1kB de EEPROM, 32kB de memoria flash ISP.\\
		
		\textit{Características:} mini USB, controlador ethernet ENC8J60, conector ethernet, LEDs de diagnóstico.\\
		
		\textit{Precio:} aprox. 50\$.\\
		
		\item \textbf{Teensy 3.6. o Teensy 2.0}. La familia de placas Teensy fue desarrollada por PJRC durante los últimos años. Inicialmente, Teensy se creó prometiendo un poco más de factor de forma más pequeño. Uno de los Teensy más apreciados es el 3.6.; tiene un procesador ARM de 32 bits, 52 entradas/salidas en total, ADC, lector de tarjetas SD y muchas más funciones. Además, PJRC proporciona un complemento software \textit{Teensyduino} para una compatibilidad casi completa con los programas Arduino. \\ 
		
		\textit{Especificaciones}: 180-MHz Cortex M4F, 256 kB RAM, almacenamiento flash 1MB, 4K EEPROM. \\
		
		\textit{Características}: Teensyduino para Arduino IDE, lector SD. \\
		
		\textit{Precio}: aprox. 30 \$ en PJRC. \\
		
		\item \textbf{Launchpad MSP430}. Launchpad es una placa de desarrollo fabricada por Texas Instruments (TI). Junto con el hecho de que viene con un MSP430 de TI, también brinda compatibilidad con el software y hardware de TI. Esto viene en forma de paquetes de refuerzo y herramientas de desarrollo en línea. Además, permite la depuración onboard. Existe un IDE similar a Arduino llamado Energia para un enfoque más sencillo. \\
		
		\textit{Especificaciones}: 24-MHz MSP430, 32 kB Program FRAM, 4 kB RAM. \\
		
		\textit{Características}: tecnología \textit{EnergyTrace} para bajo consumo, Energia (similar a Arduino IDE), dos botones onboard y LEDs, compatible con el \textit{Booster Pack}. \\
		
		\textit{Precio}: aprox. 13 \$ en Texas Instruments. \\
		
		\item \textbf{Netduino N3 Wi-Fi}. Un tipo diferente de alternativa a Arduino de Wilderness Labs, el Netduino N3 utiliza .NET MicroFramework y Netduino.Foundation Framework, lo que significa que está programado en un lenguaje de programación de nivel superior (C\#). Esto puede resultar atractivo para aquellos que no quieren lidiar con los niveles más bajos de programación de MCU.
		
		Esta placa está hecha con dispositivos inalámbricos en mente, con Wi-Fi incorporado. Otra característica notable de Netduino es la disposición de los pines hecha para la compatibilidad del shield Arduino. Dicho todo esto, Netduino puede atraer a aquellos que ya están familiarizados con el marco .NET y les gustaría programar su placa MCU a través de Microsoft Visual Studio y similares. \\
		
		\textit{Especificaciones}: 168 MHz, 164+ kB RAM, 1408kB de almacenamiento flash.  \\
		
		\textit{Características}: .NET  MicroFramework, librerías Netduino.Foundation, Wi-Fi, lector SD,  22 pines de entrada-salida de propósito general (GPIO). \\
		
		\textit{Precio}: aprox. 50\$. \\
		
		\item \textbf{Particle Photon}. Particle Photon es la placa ideal para el proyecto de IoT conectado. El Photon tiene un chip Wi-Fi Cypress integrado y viene con conectividad gratuita con Device Cloud y todas sus funciones. Particle incluso proporciona JavaScript y SDK móviles junto con su IDE web y local.
		
		Gracias a todo el soporte, guías y herramientas proporcionadas por Particle, Photon es una buena alternativa a Arduino. Es ideal para aficionados principiantes o intermedios que buscan la plataforma adecuada para un proyecto de IoT simple. \\
		
		\textit{Especificaciones}:  ARM Cortex M3 de 120 MHz, RAM de 128 kB, almacenamiento flash de 1 MB.\\
		
		\textit{Características}: Device Cloud, SDK para dispositivos móviles y ParticleJS, 18 GPIO, muchas referencias y guías. \\
		
		\textit{Precio}: aprox. 19\$.\\
		
		\item \textbf{SparkFun Thing Plus}. SparkFun tomó el ESP32 e hizo esta placa. El ESP32 es sin duda una gran opción para una placa IoT. Viene con Wi-Fi, Bluetooth y Bluetooth Low Energy integrados. Incluso agregaron una conexión Li-Po para complementar aún más su naturaleza inalámbrica.
		
		Thing Plus también muestra algunas especificaciones impresionantes y un buen conjunto de sensores. Además, es compatible con Arduino IDE. \\
		
		\textit{Especificaciones}: 240 MHz, 520 kB SRAM, 16MB de almacenamiento flash. \\
		
		\textit{Características}: Wi-Fi, Bluetooth, BLE, cargador de batería Li-Po, sensor de efecto Hall, sensor táctil capacitivo, sensor de temperatura. \\
		
		\textit{Precio}: aprox. 21\$ en SparkFun.\\
		
		\item \textbf{Adafruit Feather Huzzah}. El Huzzah proviene de la línea Feather de Adafruit. Está destinado a ser una placa pequeña. Tiene Wi-Fi incorporado, un cargador Li-Po y compatibilidad con Arduino. Además, como ESP8266, se puede utilizar con NodeMCU Lua.
		
		Puede parecer que esta placa parece una versión inferior de Thing Plus, ya que el ESP32 es esencialmente el sucesor del ESP8266. Esta placa, sin embargo, ha existido lo suficientemente presente como para que haya ganado una gran comunidad de usuarios. Esto hace que esta placa sea accesible para los aficionados que deseen la experiencia colectiva de dicha comunidad. \\
		
		\textit{Especificaciones}: 80 MHz, 50 kB RAM, 4MB de almacenamiento flash. \\
		
		\textit{Características}: Wi-Fi, cargador Li-Po, compatible con Arduino, NodeMCU Lua, 9 GPIO.\\
		
		\textit{Precio}: aprox. 17\$ en Adafruit.\\
		
		\item \textbf{BeagleBoard PocketBeagle}. El PocketBeagle es el más pequeño de los BeagleBoards. Ejecuta Linux desde el primer momento y se puede programar a través de su navegador web. Esta opción, por tanto, se trataría de un ordenador, como lo es una Raspberry Pi, no un microcontrolador. Ideal para proyectos pequeños que necesitan los recursos de una computadora. 
		
		El Sitara AM3358 ARM Cortex-A8 en el PocketBeagle tiene dos PRU (unidades programables en tiempo real) que son esencialmente microcontroladores conectados al procesador principal. Estos son, en palabras de sus preguntas frecuentes, "ideales para una latencia baja predecible, mientras que el procesador ARM es bueno para el rendimiento". \\
		
		\textit{Especificaciones}: 1GHz Cortex-A8, 512 MB RAM\\
		
		\textit{Características}: sistema operativo Linux, conector SD, 2 PRUs, 72 pines de expansión, fácilmente programable a través de un navegador web.\\
		
		\textit{Precio}: aprox. 35\$. \\
		
		\item \textbf{SparkFun RedBoard Artemis}. El RedBoard Artemis es la versión mejorada de SparkFun del Arduino Uno. Viene en el mismo factor de forma y agrega Bluetooth, 24 GPIO, 21 PWM y aumenta significativamente las especificaciones.
		
		SparkFun continúa reconociendo el valor de la plataforma Arduino y, por lo tanto, ha hecho que esta placa sea compatible con el IDE de Arduino. Además, han expuesto el conector JTAG para aquellos usuarios más avanzados que desean depurar correctamente su MCU.
		
		Esta placa relativamente nueva también es capaz de ejecutar modelos TensorFlow, llevándola al IDE de Arduino.  \\
		
		\textit{Especificaciones}: 48 MHz (96 MHz turbo), 384 kB RAM, 1MB de almacenamiento flash. \\
		
		\textit{Características}: compatibilidad con Arduino IDE, Bluetooth, 24 GPIO.\\
		
		\textit{Precio}: aprox. 20\$.\\
		
		\item \textbf{STM32F3 Discovery}. El STM32F3 es solo una de las diversas placas Discovery fabricadas por ST que vale la pena revisar. Este, sin embargo, viene con un giroscopio y acelerómetro de 3 ejes, un sensor magnético 3D, 10 LED y algunas características más.
		
		Si bien esta alternativa no tiene compatibilidad con Arduino IDE, ST proporciona herramientas de software. Actualmente, su ecosistema STM32Cube incluye un generador de código de inicialización e IDE. \\
		
		\textit{Especificaciones}: 48 kB RAM, 256kB de almacenamiento flash.\\
		
		\textit{Características}: Giroscopio/acelerómetro de 3 ejes, sensor magnético 3D, 10 LED, 2 botones, sensor de movimiento.\\
		
		\textit{Precio}: aprox. 16\$ en ST.
		\\
		
		\item \textbf{Silicon Labs Wonder Gecko}. Esta placa de Silicon Labs se basa en el EFM32. Si bien es una placa más cara, viene con una serie de características muy destacables. Incluyen un micro LCD, un control deslizante táctil, un sensor de luz ambiental y un condensador de 0.03F para el dominio de energía de respaldo, por nombrar algunos.
		
		El Wonder Gecko está más orientado al desarrollo, pero aún tiene una cantidad decente de referencias y guías para ser accesible. Silicon Labs también ofrece su Simplicity Studio. Si el precio no es un problema, esta placa y sus variaciones son buenas alternativas. \\
		
		\textit{Especificaciones}: 48MHz Cortex M4, 32 kB RAM, 256kB almacenamiento flash.\\
		
		\textit{Características}: debugger integrado, sistema de monitoreo de energía, varios sensores, LCD, Simplicity Studio.\\
		
		\textit{Precio}: aprox.  100\$ en Silicon Labs.\\
		
		\item \textbf{Raspberry Pi Pico}. Este microcontrolador lleva en el mercado desde el 25 de enero de 2021, apenas unos meses; este hecho, sin embargo, no impide que miles de personas ya lo hayan probado, y es que actualmente existe en internet muchos proyectos y documentación acerca de esta nueva Raspberry Pi, la cual se aleja de su antecesora mini ordenador para acercarse más al mundo de los microcontroladores similares a Arduino. Está construido sobre el propio silicio interno de Raspberry Pi, el RP2040. El RP2040 también forma parte de un ecosistema más amplio. Adafruit, Pimoroni, SparkFun e incluso el equipo de Arduino han desarrollado placas adaptadas a necesidades más específicas. Uno de los dispositivos de Pimoroni, el PicoSystem, es una "experiencia de creación de juegos portátil" de 58.50 £ (alrededor de 80 \$ ) que te permite programar una consola incluso más pequeña que la Game Boy Micro.\\
		
		Para los usuarios avanzados, la organización Raspberry Pi proporciona un C SDK completo, una cadena de herramientas basada en GCC e integración con Visual Studio Code. \\
		
		\textit{Especificaciones}: ARM Cortex-M0+ de doble núcleo, 264 KB de RAM y 16 MB de memoria flash (2 MB integrados). \\
		
		\textit{Características}: 30 pines GPIO, controlador USB 1.1 (más el modo de almacenamiento USB), compatibilidad con Arduino IDE.\\
		
		\textit{Precio}: 4 \$.\\
		
		\item \textbf{Opción DIY} (hacer tú mismo tu propio arduino o microcontrolador). Existen tutoriales varios en internet que nos enseñan a crear nuestro propio microcontrolador haciendo uso de componentes como ICs, diodos, resistencias, condensadores, etc. Es una buena opción para aprender cómo funciona un microcontrolador a nivel electrónico, o si queremos personalizar a nuestro gusto para reducir el consumo del sistema global que constituye el microcontrolador, por ejemplo.\\

	\end{itemize}
	\subsubsubsection{Comunicación radio}
	
	\noindent En tanto a posibles opciones para llevar a cabo una comunicación radio efectiva, existen muchas, pero la investigación se centrará en aquellas que nos puedan servir para aplicaciones IoT, las cuales se caracterizan por buscar ser autónomas (es decir, priorizar bajo consumo y ahorro de energía), de largo alcance, con transferencia de datos baja/media, seguridad y chequeo de errores, y conectividad entre redes y la nube, entre otros. Se suele hablar de redes LPWAN, para conseguir conectividad IoT, y dentro de esta categoría tenemos varias opciones, entre las que se encuentra LoRa, la cual ya fue preseleccionada para este proyecto por su soporte \textit{Open Source} y equipos asequibles para la creación de redes sencillas, aunque también complejas (Ademas de que conociamos las caracteristicas del terreno, y demas), y se deseaba hacer un estudio sobre prestaciones de equipos que usaran esta tecnología.\\
	
	\noindent \textbf{Qué es una red LPWAN} \\
	
	\noindent Si uno investiga acerca de \textit{Internet of Things}, llega a la conclusión de que existe una cantidad abrumadora de opciones para conectividad IoT: desde Wi-Fi hasta Bluetooth, NB-IoT hasta CAT-M1, y LoRa hasta RPMA. Cada aplicación IoT tiene sus propios requerimientos, lo que quiere decir que una opción de conectividad IoT que es perfecta para una aplicación puede ser una mala opción para otra. Por ejemplo, si la aplicación tiene muchos sensores remotos, la vida útil de la batería es una consideración importante, pero si la aplicación necesita enviar muchos datos lo es el ancho de banda, o si lleva involucrados datos que son vitales, lo es el tiempo. \\
	
	\noindent 
	\subsubsubsection{Comedero}
	
	\noindent \textit{Servomotor} \\
	
	\noindent Un servomotor es... \\
	
	\noindent Algunos parámetros a tener en cuenta son... \\
	\noindent Stall current. Máxima corriente que se genera cuando el motor está aplicando su máximo torque. Free current. Corriente que se genera cuando el motor está girando libremente a su máxima velocidad, sin otra carga que la propia fricción y fuerzas contralectromotrices (\textit{back-EMF}). \\
	
	\subsubsubsection{Bebedero}
	
	\noindent \textit{Sensor ultrasónico}\\
	
	\noindent \textit{Bomba de agua} \\
	
	\noindent \textit{Sensores de nivel de agua} \\
	
	\noindent \textit{Relé} \\
	
	\subsubsection{Prototipo exterior}

	% Definición funcionalidades: qué debo hacer y qué necesitaría
	% Investigación componentes (*) e integración (programación)
		%(*) Qué se ha usado, desglosado por funcionalidades: bebedero, comedero, alimentación, microcontrolador y radio.
		% Descripción, foto, foto en prototipo protoboard
	% Definición prototipo provisional: opciones que he descartado, QR de que funciona?
	% Diseño PCB
	% Desarrollo BBDD en servidor y app web para ver el estado del sistema de alimentación
	
	\subsection[Elección soluciones]{Elección de soluciones para cada funcionalidad}
	\label{Subsección 3.3: elección de soluciones para cada funcionalidad}
	
	\subsubsection{Alimentación autónoma del dispositivo}
	
	\noindent Para conseguir el objetivo de una alimentación autónoma para el dispositivo de alimentación final que queremos crear, se ha decidido usar la siguiente combinación de componentes, explicados de manera teórica y más ampliamente en el apartado \ref{subseccion 3.2.1: búsqueda soluciones alimentación}. \\
	
	\begin{itemize}
		\item Batería recargable Li-ion 18650 sin protección
		\item Kit protección 18650
		\item TP4056
		\item Panel solar 6V
	\end{itemize}
	
	\noindent Como se indica, la batería 18650 no tiene protección. Esto significa que cualquier manipulación incorrecta de la misma puede derivar en la rotura de la pila, o incluso en que ésta se inflame.  Además, como se indicó en el apartado \ref{subseccion 3.2.1: búsqueda soluciones alimentación}, dentro de \textit{opciones para baterías recargables}, se necesita de un circuito que ayude a la pila, hecha de Li-ion, a una correcta carga y descarga, precisamente porque conlleva las mismas consecuencias de una mala manipulación, expuestas anteriormente. Es por ello que se realizó una investigación en tanto a circuitos de protección para evitar sobredecarga, sobrecarga y cortocircuitos, en este tipo de pila (Li-ion). \\
	
	\noindent A continuación, se describirán brevemente los componentes enumerados y el por qué de su elección. Para más detalle, se pueden consultar los \textit{datasheet} en el Anexo I. \\
	
	\noindent \textbf{TP4056} \\
	
	%\noindent El TP4056 es un chip encapsulado en formato SOP-8 que es capaz de gestionar la carga de una batería. Es decir, adecua la entrada de energía para el estándar de 1A de la mayoría de baterías de litio que se usan en la industria electrónica, y también es capaz de controlar la temperatura. \\
	\noindent Se requiere para gestionar la carga de la batería Li-ion que vamos a utilizar.  Su funcionamiento, a grandes rasgos, consiste en transformar la fuente de energía eléctrica que tiene a su entrada para cargar la batería que tiene conectada a su salida; debido a que las baterías requieren que la energía sea cualitativa y cuantitativamente de una manera concreta, este paso es necesario. Las fuentes de energía eléctrica pueden ser diversas, desde un adaptador conectado a la red eléctrica, una fuente de alimentación, un panel solar, un generador, etc. No en todos los casos se necesitan los mismos componentes, depende de cada caso.\\
	
	\noindent Un ejemplo de TP4056 sencillo y fácil de encontrar sería el siguiente:  \\
		
	\begin{figure}[h]
		\begin{center}
			\includegraphics[width=0.6\textwidth]{img/tp4056_withoutBoost.png}
			\caption{Ejemplo de un TP4056 simple.}
			\label{TP4056 sin boost}
		\end{center}
	\end{figure}
	
	\noindent Del esquema anterior del módulo TP4056, podemos destacar: \\
	
	\begin{itemize}
		\item \textbf{Puerto mini-USB}: para alimentar tu batería a través de este tipo de cables. 
		\item \textbf{Bornas IN-, IN+}: están a los lados del puerto. Sirven también para alimentar la batería. A estas bornas podemos conectar una placa solar, u otra fuente que se necesite.
		\item \textbf{LEDs de carga y completado}: avisarán cuando la batería esté cargándose o cuando haya terminado el proceso.
		\item \textbf{BAT+ y BAT-}: son los otros bornes de salida que irán conectados a los bornes de la batería que se necesita cargar.
	\end{itemize}

	\noindent Sin embargo, se requieren más elementos para alimentar al dispositivo, como por ejemplo un conversor de tensión tras la batería (entre batería y dispositivo). En nuestro caso, lo requerimos, ya que la batería usada otorga 3.7V nominales (no existen baterías de 5V) y nosotros necesitamos 5V para alimentar el sistema de alimentación. Necesitamos un elemento que nos aumente la tensión a su salida, para alimentar de manera correcta el sistema creado; usaremos, por tanto, un \textit{boost converter}. Es un dispositivo que puede encontrarse aislado, aunque también podemos encontrarlo en una misma PCB con el TP4056, y será esta última opción la que usemos. \\
	
	\begin{figure}[h]
		\begin{center}
			\includegraphics[width=0.3\textwidth]{img/tp4056_withBoost.png}
			\caption{Ejemplo de un TP4056 con boost.}
			\label{TP4056 con boost}
		\end{center}
	\end{figure}
	
	\noindent Problema: no tiene protección contra sobredescargas. \\
	
	\noindent Existen TP4056 con protección, gracias al MOSFET DW01 (o DW01A). \\
	
	\begin{figure}[h]
		\begin{center}
			\includegraphics[width=0.4\textwidth]{img/tp4056_withProtection.png}
			\caption{Ejemplo de un TP4056 con protección.}
			\label{TP4056 con protección}
		\end{center}
	\end{figure}
	

	\subsubsection{Automatización y monitorización}
	
	\noindent Para la automatización y monitorización del sistema de alimentación, se ha elegido, en tanto a los bloques de microcontrolador y  comunicación radio, las tecnologías Arduino y LoRa, respectivamente. Si bien estas elecciones se habrían realizado a priori por ser opciones \textit{Open Source} muy extendidas y conocidas por la comunidad, no ha supuesto que no se investiguen otras, las cuales se han mencionado en el apartado \ref{subsección 3.2.2: automatización y monitorización}.
	
	\subsubsubsection{Entorno Arduino}
	
	% Intro: qué es Arduino
	
	\noindent Arduino es una plataforma de electrónica \textit{open-source} basada en hardware y software fáciles de usar. Nació en el \textit{Ivrea Interaction Design Institute} (Italia) como una herramienta fácil para la creación rápida de prototipos, dirigida a estudiantes sin experiencia en electrónica y programación. Tan pronto como llegó a una comunidad más amplia, la placa Arduino comenzó a cambiar para adaptarse a las nuevas necesidades y desafíos, tanto de usuarios principiantes como expertos. \\
	
	\noindent Las placas Arduino son capaces, por una parte, de ejecutar una serie de instrucciones (determinadas por el usuario) a través de su microcontrolador, de manera que a partir de la lectura de sus entradas analógicas/digitales se puedan generar salidas (analógicas/digitales); por otra parte, son programables a través del software Arduino, lo que se conoce como Arduino IDE. \\
	
	\noindent \textbf{Características} \\
	
	\noindent La principal característica de Arduino es la simplicidad para trabajar con microcontroladores. A raíz de esto, ofrece una serie de ventajas para sus usuarios, como son:
	\begin{itemize}
		\item \textbf{Es de bajo coste}. Las placas de Arduino son relativamente de bajo coste comparadas con las plataformas de otros microcontroladores. La versión más económica de un módulo Arduino puede ser ensamblada a mano, e incluso los módulos Arduino pre-ensamblados cuestan menos de 50 \$.
		\item \textbf{Multiplataforma}. El software de Arduino, Arduino IDE, es capaz de ejecutarse en Macintosh OSX, Windows y Linux. La mayoría de sistemas de microcontroladores están limitados a Windows.
		\item  	\textbf{Entorno de programación simple y claro}. El software de Arduino (IDE) es fácil de usar para principiantes, sin dejar de ser lo suficientemente flexible para usuarios más avanzados.
		\item \textbf{Software Open Source y extensible}. El software Arduino está publicado como un conjunto de herramientas open source, disponible para ser ampliada por otros programadores a través de librerías basadas en el lenguaje C++; las personas que quieran comprender los detalles técnicos pueden dar el salto de Arduino al lenguaje de programación AVR C en el que se basa. Del mismo modo, se puede agregar código AVR-C directamente en los  programas Arduino si así se desea.
		\item \textbf{Hardware Open Source y extensible}. Los planos de las placas Arduino están publicados bajo licencia \textit{Creative Commons}, así pues, diseñadores de circuitos experimentados pueden hacer su propia versión del módulo, extendiéndola y mejorándola. Incluso usuarios sin experiencia pueden crear una versión en protoboard del módulo para entender cómo funciona y ahorrar dinero.
	
	\end{itemize}

	\noindent \textbf{Hardware: placas Arduino} \\
	
	\noindent Existen diferentes placas Arduino, cuya elección dependerá del objetivo que persiga nuestro proyecto y de los requerimientos y limitaciones sobre dicho proyecto. \\
	
	\noindent Algunas de las placas básicas de Arduino más usadas son Arduino nano, Arduino uno o Arduino mega. Nos centraremos en el Arduino nano, ya que es el que vamos a usar; aún así, ya se comentaron las especificaciones y características del Arduino uno en el apartado \ref{subsección 3.2.2: automatización y monitorización}, sección \textit{Microcontroladores}, como ejemplo de placa Arduino de cara a comparar con otras opciones de microcontrolador, por lo que no se volverá a explicar en esta sección.  \\
	
	\noindent \textit{Especificaciones del Arduino nano} \\
	
	\begin{itemize}
		\item Microcontrolador: ATmega328
		\item Arquitectura: AVR
		\item Tensión de funcionamiento: 5V
		\item Memoria flash: 32kB de los cuales 2kB son para el gestor de arranque (\textit{bootloader}).
		\item SRAM: 2kB
		\item Velocidad del reloj: 16 MHz
		\item Pines de entrada analógicos: 8
		\item EEPROM: 1 kB
		\item Corriente CC por pin de entrada/salida: 40mA (pines I/O)
		\item Tensión de entrada: 7-12V
		\item Pines digitales de entrada/salida: 22 (de los cuales 6 son PWM)
		\item Salida PWM: 6
		\item Consumo de energía: 19 mA
		\item Tamaño de la PCB: 18 x 45 mm
		\item Peso: 7g
	\end{itemize}
	
	\noindent \textit{Pinout del Arduino nano} \\
	
	\noindent La funcionalidad de cada uno de los pines analógicos y digitales se pueden ver en la Figura \ref{Pinout Arduino nano}. \\
	
	\begin{figure}[h]
		\begin{center}
			\includegraphics[width=0.7\textwidth]{img/Pinout-NANO_official.png}
			\caption{Pinout de un Arduino nano.}
			\label{Pinout Arduino nano}
		\end{center}
	\end{figure}

	\pagebreak
	
	\noindent Tendremos la misma cantidad de pines digitales configurables como entrada o salida (D2 a D13), donde 6 de ellos tendrán capacidad PWM, igual que arduino uno. Respecto a los pines analógicos tenemos los pines A0 a A5 igual que el modelo uno, pero aparecen dos pines adicionales, A6 y A7, consiguiendo así un total de 8 entradas analógicas. \\
	
	\noindent [Foto pines digitales y analógicos, señalando los que son PWM.] \\
	
	\noindent Los dos primeros pines de la esquina inferior derecha corresponden a D1 para TX y D0 para RX de la comunicación serie por UART; en Arduino uno estos mismos pines también aparecían pero en orden invertido. Son un reflejo de las líneas de comunicación provistas por el conversor USB-TTL, aunque por lo general usamos la librería Software serial para definir otros dos pines cualquiera para dicha función. Ya que estamos hablando de comunicación serie, tenemos los pines asociados a la interfaz SPI, iguales que en el modelo uno; el pin digital 10 ejecutando la función de slave select, el 11 como MOSI, el 12 como MISO y el 13 como serial clock. Al igual que en el modelo uno, la interfaz de dos cables, I2C, estará presente en A4 para SDA y en A5 para SCL. \\
	
	\noindent [Foto pines digitales y analógicos, señalando los especiales para comunicaciones SPI, I2C y UART.] \\
		
	\noindent El resto de pines: Salida de 3,3V. Referencia de las entradas analógicas AREF. Salida regulada de 5V, usada normalmente para alimentar sensores y dispositivos, y dos pines de GND o masa en ambos lados de la placa. Un pin de RESET para realizar dicha función de forma remota si no tenemos acceso al pulsador físico de la placa. Finalmente, Vin se usará para alimentar la placa con una fuente no regulada de entre 7 y 12V; en arduino uno teníamos un conector para alimentación externa de tipo Jack, en arduino uno, como la placa tiene un tamaño reducido, no resultaría práctico colocar un conector tan voluminoso, por eso tenemos un pin destinado para alimentación externa. \\
	
	\noindent [Foto pines digitales y analógicos, señalando los especiales para comunicaciones SPI, I2C y UART, y pines de alimentación.] \\
	
	\noindent Pin de salida de 5V refleja los mismos 5V del puerto  del ordenador; si ahora deseamos trabajar de manera autónoma, sin conexión al ordenador, podremos alimentar la placa con una fuente externa con valores de tensión de entre 7 y 12 V. Dicha tensión se aplica de forma directa al regulador lineal de tensión, el cual convertirá la tensión de entrada no regulada a 5V regulados con un suministro máximo de corriente de hasta 500mA (la misma forma en la que trabaja el arduino uno cuando aplicamos una fuente externa mediante el jack de alimentación: entra una tensión entre 7-12V, pasa por el regulador, y en el pin de salida 5V tenemos los 5V regulados). Para la salida de 3.3V, es diferente en nano respecto de uno; en nano, los 3.3V son generados por el propio conversor USB-TTL (el circuito integrado lleva incorporado el regulador); el modelo uno dispone de un regulador por separado y destinado exclusivamente para el riel de 3.3V. En uno, dicho regulador puede suministrar hasta 150 mA de corriente máxima, mientras que en nano, como no tenemos un regulador especializado, y forma parte de la funcionalidad del conversor, el suministro de corriente será menor, de 50 mA como máximo. \\
	
	\noindent [Foto parte rear del nano señalando partes implicadas en la alimentación del nano.] \\
	
	\noindent \textit{Componentes que forman el Arduino nano} \\
	
	\noindent \textbf{Vista superior:} conector mini USB, más pequeño que el que usa arduino uno. El arduino nano tiene el mismo microcontrolador que el arduino uno (ATmega328P con encapsulado de montaje superficial), por lo que los programas/instrucciones que se desarrollen para uno son válidos/equivalentes para el otro. \\
	
	\noindent [Foto vista superior e inferior del Arduino nano.] \\
	
	\noindent \textbf{Vista inferior:} tenemos una tirada de pines macho para conectar a protoboard o a cables directamente. Difiere del arduino uno en este sentido, ya que el arduino uno tenía pines hembra en la parte superior. La idea principal para estos pines macho es conectarla a protoboard en un circuito impreso con un zócalo o soldado de forma directa.
	Cerca del conector USB tenemos un CI, un conversor USB-TTL, que nos permite conectar la placa al ordenador, obtener de él alimentación y comunicación para cargar programas y monitoreo serie. Finalmente, un regulador de tensión de 5V, permite obtener dicha tensión en un pin de salida para alimentar dispositivos externos cuando es alimentada la placa de forma externa mediante un pin especial llamado Vin.  \\

	\noindent \textit{Justificación de elección para el proyecto} \\
	
	\noindent Se ha elegido esta opción por una serie de razones, que a continuación se expondrán: \\
	
	\begin{itemize}
		\item Bajo consumo
		\item Precio reducido
		\item Facilidad para la incorporación a una PCB
		\item Tamaño reducido
		\item Amplia cobertura por la comunidad
		\item Open Source
	\end{itemize}
	

	\noindent Existen placas Arduino que incorporan conectividad IoT (no sólo LoRa, también WiFi, BLE, Bluetooth o Sigfox), como Arduino MKR FOX 1200, Arduino MKR WAN 1300, Arduino MKR WAN 1310, Arduino MKR NB 1500, Arduino Nano 33 BLE Sense, Arduino Uno WiFi REV2, Arduino MKR1000 WiFi, Arduino Nano 33 BLE, Arduino Nano 33 IoT, Arduino Nano 33 BLE, o Arduino MKR WiFi 1010.  Al ser soluciones ya integradas y muy concretas, orientadas más a redes LPWAN con conectividad a internet, no se han considerado como solución para este proyecto, pero se mencionan ya que se han estudiado y pueden resultar útiles para futuras investigaciones o proyectos similares.\\
	
	% Electrónica: arduino nano
	
	\noindent \textbf{Software: Arduino IDE} \\
	
	% Programación: arduinoIDE, void loop y setup, define, etc
	
	\noindent Como hemos mencionado anteriormente, las placas Arduino se pueden programar a través del software Arduino IDE. Además, A continuación, se explicarán los fundamentos básicos que se requieren para empezar a usar este software. \\
	
	\begin{itemize}
		\item Bloque de definición de variables globales.
		\item Bloque de definición de las tareas a realizar por el microcontrolador.
	\end{itemize}
	
	\subsubsubsection{LoRa}
	
	\noindent \textbf{Qué es}\\
	
	\noindent El término LoRa proviene de la unión de las palabras \textit{Long} y \textit{Range}, es decir, largo alcance, evidenciando, así, a través de su nombre una de sus principales características. Se trata de la capa física del protocolo \textit{LoRaWAN}; es a través de esta capa que se permite el establecimiento del enlace de comunicación de largo alcance, mientras que LoRaWAN define el protocolo de comunicación y la arquitectura del sistema para la red. \\
	
	\noindent También se trata de una tecnología inalámbrica propiamente dicha que ofrece una transmisión de largo alcance, bajo consumo y segura para aplicaciones IoT y M2M. Se basa en la modulación de espectro ensanchado, la cual presenta características de bajo consumo como la modulación FSK pero puede ser usada para comunicaciones de largo alcance. Puede ser usada para conectar sensores, gateways, máquinas, dispositivos, animales, personas, etc. de manera inalámbrica a la nube.\\
	
	\noindent \textbf{Planes de frecuencia} \\
	
	\noindent La tecnología LoRa opera en diferentes frecuencias dependiendo de la región en la que nos encontremos, por ejemplo: \\
	
	\begin{itemize}
		\item En Norteamérica, opera en la banda de 915 MHz.
		\item En Europa, en la banda de 868 MHz.
		\item En Asia, en la banda de 865 a 867 MHz y en la banda 920 a 923 MHz, aunque es común también la banda 433 MHz.
	\end{itemize}
	
	\noindent Dado que los planes de frecuencia para LoRa varían según la región, inclusive país, es recomendable asegurarse de las regulaciones vigentes en cada país para el uso del espacio radioeléctrico. A modo resumen y como orientación inicial a este asunto, se puede consultar esta \href{https://www.thethingsnetwork.org/docs/lorawan/frequencies-by-country/index.html}{lista de planes de frecuencia para cada país}, elaborada por \textit{The Things Network}. \\
	
	\noindent \textbf{Origen} \\
	
	\noindent La tecnología LoRa fue creada por una compañía francesa llamada \textit{Cycleo}, la cual posteriormente fue adquirida por Semtech en 2012. Semtech fue el miembro fundador de \textit{LoRa Alliance}, que actualmente es el organismo rector de LoRa Technology. LoRa Alliance es una de las alianzas tecnológicas de más rápido crecimiento. Esta asociación sin ánimo de lucro está formada por más de 500 empresas miembro, comprometidas a permitir la implementación a gran escala de redes IoT de amplia cobertura y baja potencia (LPWAN) a través del desarrollo y la promoción del estándar abierto LoRaWAN.\\
	
	\noindent \textbf{Especificaciones en términos generales de la tecnología LoRa}
	
	\begin{itemize}
		\item Órgano rector: LoRa Alliance
		\item Estándar: 801.15.4g
		\item Frecuencia: banda ISM, 868/916 MHz
		\item Alcance: hasta 5 km (urbano) y hasta 15 km (rural)
		\item Datarate: 27 kbps
		\item Modulación: espectro ensanchado basado en la tecnología de modulación FM
		\item Seguridad: CRC de 32 bits
	\end{itemize}
	 
	 %LoRaWAN is a Low Power, Wide Area (LPWA) networking protocol developed by the LoRa Alliance, that wirelessly connects battery operated ‘things’ to the internet in regional, national or global networks, targeting key Internet of Things (IoT) requirements such as bi-directional communication, end-to-end security, mobility and localization services.
	 
	% LoRaWAN uses unlicensed spectrum in the ISM bands to define the communication protocol and system architecture for the network while the LoRa physical layer creates the long-range communication links between remote sensors and gateways connected to the network. This protocol helps in the quick setup of public or private IoT networks anywhere using hardware and software.
	
	\noindent  \textbf{Regulación LoRa en Europa: ERC 70-03} \\
	
	\noindent \href{https://docdb.cept.org/download/25c41779-cd6e/Rec7003e.pdf}{ERC 70-03} \\
	
	%% CAPA FISICA DE LORAWAN %%
	
	\pagebreak
	
	%\subsubsubsection{LoRaWAN: NO VA AQUÍ, VA EN APARTADO 3.2}
	
	%\noindent A pesar de que no se va a usar LoRaWAN en este proyecto, sí que se realizó su pertinente investigación, por lo que a continuación se muestra un resumen de nociones básicas sobre LoRaWAN. \\
	
	%\noindent \textbf{Clases de nodos finales LoRaWAN}  \\
	
	%\noindent Existe 3 clases diferentes de nodos finales en LoRaWAN, los cuales son:
	
	%\begin{itemize}
		%\item Clase A (\textit{All})
		%\item Clase B (\textit{Beacon})
		%\item Clase C (\textit{Continuous})
	%\end{itemize}
	
	%\begin{figure}[h]
		%\begin{center}
			%\includegraphics[width=0.7\textwidth]{img/endDevices.png}
			%\caption{Funcionamiento de nodos finales usando LoRaWAN.}
		%\end{center}
	%\end{figure}
	
	%\pagebreak
	
	%\noindent \textbf{Estructura de un paquete LoRaWAN}
	
	%\begin{figure}[h]
	%\begin{center}
			%\includegraphics[width=0.7\textwidth]{img/lora_phyLayer_packetFormat.png}
			%\caption{Estructura general de un paquete LoRaWAN.}
	%\end{center}
	%\end{figure}
	

	
	%\begin{figure}[h]
	%\begin{center}
			%\includegraphics[width=0.7\textwidth]{img/lora_phyLayer_packetFormat_enh.png}
			%\caption{Estructura detallada de un paquete LoRaWAN.}
	%\end{center}
	%\end{figure}

	%\pagebreak

%	\subsection[Monitorización y automatización]{Monitorización y automatización}
	
%	El Arduino que vamos a usar es el Arduino nano, como ya se ha comentado anteriormente. Los sensores que vamos a usar son:
	
%	\pagebreak
%	\subsection[Comunicaciones LoRa]{Comunicaciones LoRa}

%	Los módulos LoRa que vamos a usar son los E32-868T30D (como ya se ha comentado anteriormente).
%	\subsubsection{Cálculo teórico de pérdidas de enlace}
%	\subsubsection{Simulación en \textit{Radio Mobile}}
	
%	\subsection{Alimentación del dispositivo}

	\subsubsubsection{Comedero}
	
	\noindent Tanto en este apartado, como en el siguiente sobre elección de soluciones para el bebedero, se mostrarán los sensores y actuadores, así como creaciones con impresora 3D para llevar a cabo sus funcionalidades asociadas.\\
	
	\noindent Para llevar a cabo la función de comedero, el sistema de alimentación creado debe ser capaz, en resumen, de mover el pienso de la reserva al comedero. Para ello, debemos tener en cuenta: 
	
	\begin{itemize}
		\item La capacidad de la reserva de pienso, y del comedero.
		\item Si la reserva quedará fija o móvil, y su soporte en todo caso.
		\item Conexionado entre reserva y comedero.
		\item Cómo se gestionará el movimiento entre el pienso y el comedero; principalmente, las opciones más directas son movimiento basado en la gravedad, o movimiento gestionado por un motor. También será necesario estudiar el camino que realizará el pienso, para establecer conductos por donde pueda fluir hasta el comedero sin gran dificultad.
		\item Soporte de los conductos que reflejan el conexionado entre reserva y comedero.
		\item Conexionado eléctrico con el microcontrolador para recibir las señales de movimiento, en el caso de usar servo, o sensorización de niveles de pienso en reserva y/o en comedero.
	\end{itemize}
	
	\noindent Teniendo presente los puntos recién expuestos, se ha elegido como gestor del movimiento del pienso el uso de un motor, en concreto, de un servomotor de 5V, capaz de girar 360º y capaz de mover cargas de hasta 10 kg (para leer información más detallada, ir a Anexo I FALTA REFERENCIA) \\
	
	\noindent [Foto del servo] \\
	
	\noindent Este servo, con la ayuda de unas hélices imprimidas en 3D, moverá el pienso en dirección longitudinal a una pieza T de PVC donde irán insertados. \\
	
	\noindent [Foto] \\
	
	\noindent Además, se hará uso de una serie de piezas para crear la conexión comedero-reserva, desde la salida de pienso en el fondo de la reserva hasta la salida del pienso hacia el comedero. \\
	
	\noindent [Foto, quizás esquemático simplificado, la versión completa está en el apartado de prototipo final] \\
	
	\noindent Por último, las conexiones con el microcontrolador se realizarán a través de los cables que accionan el servo: uno para señal, otro para alimentación, y otro para tierra. \\
	
	\pagebreak
	% servo_connections_v2_fritzing.png
	\noindent [Foto cables servo y fritzing de conexionado con arduino.]\\
	
	\begin{figure}[h]
		\begin{center}
			\includegraphics[width=0.7\textwidth]{img/servo_connections_v2_fritzing.png}
			\caption{Esquema de las conexiones servo-Arduino realizado en \textit{Fritzing}.}
			\label{Esquema conexiones servo Fritzing}
		\end{center}
	\end{figure}
	
	\pagebreak
	
	\subsubsubsection{Bebedero}
	
	\noindent Opciones: ultrasónico en la tapa de la reserva + bomba + relé, sensores de nivel de agua (flotación) + relé + bomba. [Introducción en el apartado de Búsqueda de soluciones, pero todavía no está ordenado]\\
	
	\noindent Elección: sensores de nivel de flotación de agua, relé y bomba. \\
	
	\noindent Sensores de nivel de flotación de agua: se comportan como switches, y su respuesta son valores lógicos (0 o 1). \\
	
	 \noindent [Foto sensores de flotación.] \\
	 
	 \noindent Su funcionamiento es sencillo: si el nivel del agua queda por encima del sensor, el valor lógico asociado que enviará al microcontrolador será de 1. Si por el contrario queda por debajo, mandará un 0. Dos cables de alimentación, simétricos, se requiere de resistencias de 10k en configuración \textit{pull-down}, ya que al tratarse de switches, al conmutar si no existe esa resistencia da un valor lógico o desconocido o falso 1, consecuencia del ruido del pin que se introduce al estar flotante.\\
	
	\begin{figure}[h]
		\begin{center}
			\includegraphics[width=0.2\textwidth]{img/pull-down-resistor.png}
			\caption{Esquema de un switch conectado a un microcontrolador con una resistencia \textit{pull-down}.}
			\label{Esquema resistencia pull-down}
		\end{center}
	\end{figure}
	
	% bebedero_test_fritzing.jpg
	
	\pagebreak
	
	\begin{figure}[h]
		\begin{center}
			\includegraphics[width=0.7\textwidth]{img/bebedero_test_fritzing.jpg}
			\caption{Esquema de las conexiones de sensores y actuadores usados para el bebedero con Arduino realizado en \textit{Fritzing}.}
			\label{Esquema conexiones bebedero Fritzing}
		\end{center}
	\end{figure}

	\noindent La bomba se comporta como un motor DC, y tiene dos cables únicamente (+5V, GND), por lo que, para poder controlar su funcionamiento, necesitaremos usar un actuador. Elegimos un relé, el cual está conectado al pin D7 del arduino, según podemos ver en la Figura \ref{Esquema conexiones bebedero Fritzing}. Entre el pin D7 y el relé, vemos que existe una serie de componentes intermedios (un LED, una resistencia de un 1k$\Omega$). Se usan debido a que el relé necesita ver a su entrada una corriente superior, y el transistor BJT amplifica dicha corriente para su correcto funcionamiento. La resistencia sirve para estabilizar la corriente a la entrada del LED, el cual usamos para ver visualmente cuándo cambia de estado el relé (aunque cuando lo hace, se oye). \\
	
	\noindent [Foto bomba de agua usada.] \\
	
	\noindent El relé actúa como un conmutador. [Seguir explicando]. \\
	
	\noindent [Foto relé usado.] \\
	
	\subsubsection{Prototipo exterior}
		
	\subsection[Resumen del capítulo]{Resumen de capítulo}
	
	En este capítulo se ha hablado de [...]. Se sacan conclusiones para la creación de un prototipo que pueda ser ejecutado y probado en el entorno objetivo.
	\pagebreak
	
	
	\section[Prototipo y pruebas]{Prototipo y pruebas}
	
	% Ensamblado PCB y pruebas
	\subsection[Ubicación]{Ubicación}
	\subsection[Prototipo inicial]{Prototipo inicial}
	
	\noindent [Realmente no sé si unificar la subsección 4.2 y la 4.3, ya que el prototipo está bastante explicado en diseño del sistema, y aquí mi idea era explicar lo que no se haya explicado antes y la PCB, las pruebas realizadas y resultados. Siento que el prototipo inicial ya se expuso en la subsección 3.3., más conceptual y en protoboard, y el definitivo es con PCB y sumándole la alimentación autónoma. ¿Cómo lo ves tú?] \\
	
	\noindent [Anotaciones: usamos una versión diferente a la de Arduino nano vendida por la organización Arduino. Seguramente lo mueva de aquí.] \\
	
	\noindent Usaremos un arduino nano que usa un ATmega328P y un chip CH340. En la sección X se han explicado las especificaciones técnicas, pinout y componentes que forman parte del arduino nano fabricado por la organización Arduino, pero el que va a ser usado lo está por otro fabricante. A efectos prácticos, las diferencias son pocas, pero hay que considerarlas de igual manera:
	
	\begin{itemize}
		\item El microcontrolador no es ATmega328 (usado en el nano de la organización Arduino), sino que es ATmega328P. La diferencia reside en que es una versión posterior diseñada para consumir menos potencia que el 328, esencialmente tienen la misma arquitectura y características en tanto a memoria flash, EEPROM y SRAM (la P viene de \textit{picoPower}).
		\item Este arduino usa un regulador diferente, el CH340G, frente al FT232 que usa el arduino original. Además, el fusible/diodo Schottky que usan para proteger la comunicación serial frente a sobrecorrientes es diferente, siendo el MBR0520 para el caso de los arduinos originales, frente al B5819W que usa el arduino que vamos a usar.
	\end{itemize}

	\noindent Como hemos visto en la sección \ref{Sección 3: Diseño del sistema}, subsección \ref{Subsección 3.3: elección de soluciones para cada funcionalidad}, son muchos componentes los que forman el cómputo total del sistema de alimentación. Para facilitar el prototipado y simplificar las conexiones de todos los elementos, se creó una PCB, como la que se muestra a continuación: \\
	
	\noindent [Foto PCB v1.1a] \\
	
	\noindent Se adjunta en el Anexo I (FALTA REFERENCIAR) una imagen del esquemático y de la PCB que se mandó realizar. El fabricante de la PCB es JLCPCB, y se usó su editor EasyEDA. \\
	
	\subsection[Prototipo definitivo]{Prototipo definitivo}
	
	\pagebreak
	
		\begin{figure}[h]
		\begin{center}
			\includegraphics[width=0.7\textwidth]{img/conexiones tfg-comedero.png}
			\caption{}
			\label{Conexiones TFG comedero.}
		\end{center}
	\end{figure}
	
	\noindent Nótese en la Figura \ref{Conexiones TFG comedero.} que se omite la presencia de la segunda PCB que recibe los datos de monitorización del pienso.  Es necesario calibrar en software el arduino que monitorizará el pienso; esto se consigue calculando el número de recargas que ponemos realizar dada la capacidad de la reserva y del comedero. En nuestro caso, tenemos una reserva de 5 kg y un comedero de 200g, por lo que, haciendo cálculos, tenemos para 25 ciclos de rellenado. Por seguridad y por desnivel de la reserva (la pieza por donde cae el pienso está a un nivel superior del tope inferior de la reserva, por lo que esos dos centímetros de desnivel no permitirían al pienso restante caer por él), reduciremos esos ciclos al 20\%, por lo que realmente consideraremos que serán 20 ciclos de rellenado. En el script de Arduino que se adjunta en el Anexo I (REFERENCIAR)  se muestran las variables que permiten ajustar este cálculo y permiten la calibración si se cambiara la reserva o el comedero. Tendremos un contador que permitirá saber en qué ciclo de rellenado estamos, si quedan recargas o necesitamos rellenar la reserva. Esta información se transmitirá desde el transceptor LoRa del refugio hasta el transceptor LoRa en el otro extremo, y será gestionado por la Raspberry Pi al que está conectada la PCB de este extremo.\\ 
	
	\noindent Ajustar el tiempo de un ciclo de rellenado para desactivar el movimiento del servomotor y que se rellene el comedero la cantidad deseada. \\
	
	\pagebreak
	
		\begin{figure}[h]
		\begin{center}
			\includegraphics[width=0.7\textwidth]{img/conexiones tfg-bebedero.png}
			\caption{Esquema general del bebedero. }
			\label{Conexiones TFG bebedero.}
		\end{center}
	\end{figure}
	
	\noindent Nótese en la Figura \ref{Conexiones TFG bebedero.} que se omite la presencia de la segunda PCB que recibe los niveles de sensorización de agua: se miden niveles de agua con los sensores, y controlamos la acción de la bomba con el relé, conectados al arduino, el cual envía datos mediante transceptor LoRa al transceptor LoRa que tenemos en el otro extremo, el cual no necesitará alimentación autónoma al estar en una casa, y es este el que irá conectado a la Raspberry (que tiene función de servidor y gestor de los datos en recepción). \\
	
	\pagebreak
	
	\subsection[Pruebas]{Pruebas}
	
	\pagebreak
	
	\subsection[Comentarios sobre los resultados de las pruebas]{Comentarios sobre los resultados de las pruebas}
	
	\pagebreak
	
	\subsection[Presupuesto]{Presupuesto}
	\noindent Se adjunta una tabla para el desglose del presupuesto para este proyecto en el Anexo III (\ref{subsection: presupuesto}).
	
	\pagebreak
	
	\subsection[Resumen del capítulo]{Resumen del capítulo}
	
	\pagebreak
	
	\section[Conclusiones y líneas futuras]{Conclusiones y líneas futuras}
	
	\noindent \textbf{Alimentación:} estudiaría las baterías LiPo, incluso LiFePO4, que aunque son más caras, no limitan tanto en el tema de la protección. Las baterías Li ion tienen bastante cobertura y tienen sus ventajas, pero también sus desventajas. Investigaría más profundamente en general el tema de la alimentación autónoma, ya que es un mundo muy extenso que debe tratarse de manera externa a la parte radio o a la parte de microcontrolador; podemos realizar proyectos donde abarquemos todo, pero para llegar a conclusiones donde se lleguen a prototipos iniciales, que en posteriores investigaciones se expandan y mejoren sus funcionalidades de manera individual, sin perder de vista el proyecto global. Diseñaría una PCB sólo para alimentación, ya que los requerimientos de alimentación para cada proyecto son diferentes y para un proyecto de dimensión media-grande es mucho más práctico diseñar un sistema personalizado de alimentación de dispositivo, algo que conseguimos mediante una PCB, compactando los sistemas que, sin PCB, se traducen en numerosas PCBs ya fabricadas que quizá no se adapten del todo a lo que necesitamos. \\
	
	\noindent \textbf{Microcontrolador}: el arduino nano para principiantes, con previas nociones de electrónica y con previsiones de realizar una PCB, siempre y cuando el proyecto no sea muy extenso, puede ser empleado. Sin embargo, yo he experimentado que en espacio de memoria se me quedaba corto, sobretodo al usar librerías como OLED, las cuales debía excluir si quería cargar en el arduino mi código restante; teniendo en cuenta que debía importar librerías obligatoriamente para usar por ejemplo la OLED, el servomotor o el transceptor LoRa, este es un punto muy negativo, ya que es un código que no puede ser reducido. Tuve que buscar trucos para compactar el uso de dicha memoria debido a estas limitaciones. Probaría con otras opciones mencionadas en la sección \ref{subsección 3.2.2: automatización y monitorización}, donde hablamos de opciones de microcontroladores, como la Adafruit, NodeMCU o Teensy. \\
	
	\pagebreak
	
	\section*{Glosario}
	\addcontentsline{toc}{section}{Glosario}
	
	%%% MICROCONTROLADOR %%%
	\noindent \textbf{\large SRAM}. \\
	\noindent \textbf{\large EEPROM}. \\
	\noindent \textbf{\large Flash memory}. \\
	\noindent \textbf{\large Bootloader.} \\
	
	%%% SERVOMOTOR %%%
	\noindent \textbf{\large Torque} (en un motor), o par. En mecánica newtoniana, se denomina momento de una fuerza o torque a una magnitud vectorial, obtenida como producto vectorial del vector de posición del punto de aplicación de la fuerza por el vector fuerza, en ese orden. Para un motor, esto es, la fuerza que puede aplicar a través del movimiento rotacional de su eje para levantar o mover una carga; es importante el contexto mecánico previo que hemos establecido para entender la importancia de la dirección en la que apliquemos el movimiento del servo y la fuerza en sí.\\
	
	\noindent \textbf{\large Fuerza contraelectromotriz} (\textit{back EMF}). Fuerza electromagnética que aparece en un circuito inductivo en una dirección tal que se opone a cualquier cambio de corriente en el circuito. De hecho, la conversión de energía en un motor DC es causada por esta fuerza: la dirección de la corriente en la armadura conductiva que tienen los motores es contraria a la dirección del efecto contraelectromotriz, por lo que debemos realizar un trabajo eléctrico para contrarrestar ese efecto (usar la corriente en contra del \textit{back EMF}), el cual se traduce en realizar un trabajo mecánico, el cual es la función principal de un motor DC. Además, el \textit{back EMF} hace que los motores se autorregulen; por ejemplo, si no existe carga, se hace menos torque, se induce menor corriente y por tanto la fuerza contraelectromotriz aumenta (sería equivalente a la tensión suministrada, ya que su efecto se mide en voltios), pero si aumentamos la carga, la velocidad disminuye, decrece el \textit{back EMF}, aumenta la corriente y la magnitud del torque. Existen fórmulas que relacionan estas variables.\\
	
	%%% SWITCHES DE NIVEL DE AGUA %%%
	\noindent \textbf{\large Resistencia \textit{pull-down}}. Las resistencias \textit{pull-down}, y \textit{pull-up}, son resistencias que se usan en circuitos lógicos para asegurar un nivel lógico bien definido en un pin bajo cualquier circunstancia. Como recordatorio, los circuitos lógicos digitales tienen tres estados lógicos: alto, bajo y flotante (alta impedancia). El estado de alta impedancia ocurre cuando un pin no llega a estar en valor alto o bajo, y se queda "flotando". Una buena manera de ilustrar esta situación es un pin de entrada desconectado de un microcontrolador; no se encuentra en un estado lógico alto o bajo, y el MCU podría interpretar de manera impredecible el valor de entrada como alto o bajo. También ocurre si en un pin del MCU tenemos un switch; si no hay una resistencia \textit{pull-down} o \textit{pull-up}, el pin estará flotando cuando el switch se abra, y dando valores lógicos conocidos sólo cuanto éste se cierre. Son resistencias de valor fijo, que dependiendo de la aplicación del circuito digital que estemos diseñando tendrán un valor u otro (y de la impedancia de entrada del pin del MCU al que queramos conectarnos, por ejemplo). La diferencia entre una resistencia \textit{pull-down} y una \textit{pull-up} es que la primera se coloca entre tierra y el pin en cuestión, haciendo que, cuando el switch se abra, el pin tenga valor lógico bajo, mientras que la segunda se coloca entre VCC y el pin, haciendo que, cuando el switch se abra, el pin tenga valor lógico de alto.\\ 
	
	\pagebreak
	
	\section*{Bibliografía}
	\addcontentsline{toc}{section}{Bibliografía}
	
	\pagebreak
	
	\section*{Anexos}
	\addcontentsline{toc}{section}{Anexos}
	
	\pagebreak
	
	\subsection*{Anexo I}
	\addcontentsline{toc}{subsection}{Anexo I}	
	
	\pagebreak
	
	\subsection*{Anexo II}
	\addcontentsline{toc}{subsection}{Anexo II}
	
	\pagebreak
	
	\subsection*{Anexo III}
	\label{subsection: presupuesto}
	\addcontentsline{toc}{subsection}{Anexo III}
	
	\pagebreak
	
\end{document}
\documentclass[12pt]{article}
\usepackage[spanish, english, es-tabla]{babel}
\usepackage[utf8]{inputenc}
\usepackage{amsmath,amssymb}
\usepackage{graphicx}

\usepackage{array}

\usepackage{hyperref}
\usepackage{subcaption}
\usepackage[left = 2cm, right = 2cm, bottom = 2cm, top = 3cm]{geometry}

\hypersetup{
	pdftitle= {Diseño y automatización de sistema de alimentación para animales a través de la tecnología LoRa},
	pdfauthor = {Lucía Francoso Fernández},
	pdfsubject = {Electrónica y comunicaciones móviles},
	pdfkeywords = {Arduino, LoRa, automatización, LoRaWAN, punto a punto, OpenSource}
}

\begin{document}
	\selectlanguage{spanish}
	\title{Diseño y automatización de sistema de alimentación para animales a través de la tecnología LoRa}
	\author{Lucía Francoso Fernández}
	\date{Marzo 2021}
	
	\maketitle
	\pagebreak
	
	\tableofcontents
	
	\pagebreak

	\listoffigures
	\addcontentsline{toc}{section}{Índice de figuras}

	\listoftables
	\addcontentsline{toc}{section}{Índice de tablas}
	
	\section[Introducción]{Introducción} %corchetes, marcapaginas, llaves, texto
	
	\subsection[Contexto y justificación del trabajo]{Contexto y justificación del trabajo}
	\subsubsection[Ejemplo de caso de aplicación]{Ejemplo de caso de aplicación}
	\subsection[Objetivos del trabajo]{Objetivos del trabajo}
	\subsection[Enfoque y método seguido]{Enfoque y método seguido}
	\subsection[Planificación del trabajo]{Planificación del trabajo}
	\subsubsection[Alcance]{Alcance}
	\subsubsection[Hitos]{Hitos}	
	\subsubsection[Calendario de trabajo]{Calendario de trabajo}
	\subsubsection[Tareas y diagrama de Gantt]{Tareas y diagrama de Gantt}
	\subsubsection[Riesgos e incidencias]{Riesgos e incidencias}
	\subsubsection[Recursos]{Recursos}
	\subsection[Breve sumario de productos obtenidos]{Breve sumario de productos obtenidos}
	\subsection[Breve descripción de los capítulos restantes de la memoria]{Breve descripción de los capítulos restantes de la memoria}
	
	\pagebreak
	
	\section[Estado del arte]{Estado del arte}  
	
	\subsection[Contexto actual]{Contexto actual}
	
	\subsection[Trabajos relacionados]{Trabajos relacionados}
	\subsection[Resumen del capítulo]{Resumen del capítulo}
	
	\pagebreak
	
	\section[Diseño del sistema]{Diseño del sistema}
	\subsection[Tecnologías necesarias]{Tecnologías necesarias}
	\subsubsection[Entorno Arduino]{Entorno Arduino}
	\subsubsection[LoRa]{LoRa}
	\subsection[Monitorización y automatización]{Monitorización y automatización}
	\subsection[Comunicaciones LoRa]{Comunicaciones LoRa}
	\subsection[LoRaWAN y TTN]{LoRaWAN y TTN}
	\subsection[Resumen del capítulo]{Resumen de capítulo}
	
	\pagebreak
	
	\section[Prototipo y pruebas]{Prototipo y pruebas}
	\subsection[Ubicación]{Ubicación}
	\subsection[Prototipo inicial]{Prototipo inicial}
	\subsection[Prototipo definitivo]{Prototipo definitivo}
	\subsection[Pruebas]{Pruebas}
	\subsection[Comentarios sobre los resultados de las pruebas]{Comentarios sobre los resultados de las pruebas}
	\subsection[Presupuesto]{Presupuesto}
	\subsection[Resumen del capítulo]{Resumen del capítulo}
	
	\section[Mejora del sistema: alarmas]{Mejora del sistema: alarmas}
	\subsection[Contexto y requisitos]{Contexto y requisitos}
	\subsection[Implementación en el sistema y pruebas]{Implementación en el sistema y pruebas}
	\subsection[Pruebas con el prototipo en ubicación final]{Pruebas con el prototipo en ubicación final}
	\subsection[Resumen del capítulo]{Resumen del capítulo}
	
	\section[Conclusiones y líneas futuras]{Conclusiones y líneas futuras}
	
	\section*{Glosario}
	\addcontentsline{toc}{section}{Glosario}
	
	\section*{Bibliografía}
	\addcontentsline{toc}{section}{Bibliografía}
	
	\section*{Anexos}
	\addcontentsline{toc}{section}{Anexos}
	
	\subsection*{Anexo I}
	\addcontentsline{toc}{subsection}{Anexo I}	
	
	\subsection*{Anexo II}
	\addcontentsline{toc}{subsection}{Anexo II}
	
\end{document}